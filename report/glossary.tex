\usepackage[nonumberlist, toc]{glossaries}
\makeglossaries

\newglossaryentry{gamification}{name=gamification,
description={is the use of game thinking and game mechanics in non-game contexts to engage users in solving problems. Further descriptions about this subject may be found in reference~\cite{gamification}}}


\newglossaryentry{app}{name=mobile app,
description={is a computer program designed to run on smartphones, tablet computers and other mobile devices. The term ''app'' is a shortening of the term ''application software''
}}

\newglossaryentry{JSON}{name=JSON, 
description={is textual representation of the objects and standard way to store and transfer data between different components.}}

\newglossaryentry{synergy}{name=synergizes,
description={is a term used to describe how different elements interacts and functions together.}}

\newglossaryentry{unit}{name={unit testing},
description={is a form of testing where individual parts of the program is tested against predefined data. This makes sure future changes does not alter existing functionality unknowingly. Unit testing is a form of white box testing where the tests are written for the source code.}}

\newglossaryentry{usecase}{
name={use case},
description={is a list of steps, defining interactions between different roles using a software system and a software system. The roles can be users or external systems.}, 
plural={use cases}}

\newglossaryentry{CRUD}{name=CRUD,
description={is an acronym for create, read, update and delete.}}

\newglossaryentry{URI}{name=URI,
description={An acronym for uniform resource identifier. Is a string of characters pointing to a specific resource.}}

\newglossaryentry{DAO}{name={Data access object}, description={Data access object (DAO) is an object that provides an abstract interface to a database or other persistent storage. By mapping application method calls to persistent storage.}}
