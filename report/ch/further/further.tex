\chapter{Further development}
\label{sec:further}
\section{}
\todo[inline]{Hardware solution}
In the very beginning of the project, the team looked into the possiblity of getting live data directly from devices. This required some technical equipment for collecting and transmitting the usage of a device. The team were not able to find any cheap and easy way of doing this, so the idea was left. The concept was that the application should get live data directly form all of the devices in a house. It already exists hardware solutions that measures usage per device, but not that transmits that data wirelessly or via bluetooth. If a hardware device doing both of these things is found, it is possible to modify Wattitude to get data from the hardware device. The next step would be that the devices transmits their energy usage directly by for instance bluetooth. 

\subsection{The Cossmic projects API}
Late in the project period, the team was made aware of an API that could be used to get data from devices. The team did not have time to implement together with or test this API in any way. But, to our understanding, this API provides data from devices, whether this is actual devices or dummy data, is not known. 

\section{Energy Production}

\section{Gamification}
From the start, the customer uttered a wish of making the act of saving energy around your house, a contest of some sort. Through the concept of gamification. This was discussed within the team and a few different ideas came up. The general thought was that the user should feel that he achieved something, through saving energy and logging it within the application, and therefore would use the application to a greater extent. 

\subsection{Achievements}
One of the ideas that were discussed was the concept of achievement. That the user would get some sort of reward for achieving a saving goal, or conduct a saving tip in his residence. One of the ideas that came up was build on the achievement system that you find in the app "Untapped." \todo{Add reference to site}
This idea is based on a large base of "badges," or achievements that the user will get after doing something relevant and logging it within the application. 

Another achievement system that was discussed was to integrate different "Wattitude levels" into the application. The idea was to have one level per devotion to saving energy around your house. Say that the user has installed Wattitude \todo[inline]{Referance to each time it is mentioned?}, that gets him to the first level because he wants to be aware of his power consumption. And then, the idea moves on to several levels, for instance is level three to produce some energy yourself, to have tried at least five saving tips and to have shared your power consumtion with your friends on Facebook. \todo[inline]{Add referance} Then, the user can say that his next goal is to reach "Wattitude level 3." \todo[inline]{Not very happy about this section.}

\subsection{Wishlist}

\section{Residences}
Maa kobles til alt

\section{Profile}


\section{General}