\section{Hardware components}
In the very beginning of the project, the team looked into the possibility of getting live data directly from devices. This required some technical equipment for collecting and transmitting the usage of a device. The team were not able to find any cheap and easy way of doing this, so the idea was left. The concept was that the app should get live data directly form all of the devices in a house. It already exists hardware solutions that measures usage per device, but not that transmits that data wirelessly or via Bluetooth. If a hardware device doing both of these things is found, it is possible to modify Wattitude to get data from the hardware device. The next step would be that the devices transmits their energy usage directly by for instance Bluetooth. 

\subsection{Base station}
To ensure full operability 24 hours a day, a base station is needed in the user's home. This server will serve as an aggregating agent for data from measuring devices spread around the residence. Given an ideal architecture implementation, this unit would also be responsible for controlling devices. The team envisions this unit as a REST service mostly running synchronizations towards the Wattitude cloud server. The software for such a server could be based on the current server software. The ideal hardware for a this base station would be a Raspberry pi. These computers are very cheap, and provides all that is needed in a small board that can be hidden away. Modifications might be needed to interface with the measuring devices. 

\subsection{Measuring device}

