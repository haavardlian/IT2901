\section{New concepts}
In this section all ideas for new concepts that the team came up with are described. In essence this is functionality that has not been implemented.

\subsection{Energy Production}
The team has implemented the possibility of creating a device with the category production. This was meant to be a category where all of a residence's energy generating devices could be added. The customer wanted energy production to be in focus, and wanted the possibility of the users to not only monitor their consumption, but their energy production as well. 

It is also possible to implement a way for the user to monitor his usage contra his production, within the usage tab. By doing this the user has the possibility to see whether he is capable of selling power to his neighbours. 

\subsection{Gamification}
From the start, the customer uttered a wish of making the act of saving energy around your house a contest of some sort, through the concept of gamification. This was discussed within the team and a few different ideas came up. The general thought was that the user should feel that he achieved something by saving energy and logging it within the app, and therefore would use the app to a greater extent. 

\subsection{Achievements}
One of the ideas that were discussed was the concept of having an achievement system. The idea behind this is that the user would get some sort of reward for achieving a saving goal or implementing a saving tip in his residence. One of the ideas that came up was built on the achievement system that you find in the app "Untappd."~\cite{untappd}
This idea is based on a large base of "badges," or achievements that the user will get after doing something relevant and logging it within the app. 

Another achievement system that was discussed was to integrate different "Wattitude levels" into the app. The idea was to have one level per devotion to saving energy around your house. Installing Wattitude would get the user to level one since it indicates that he wishes to be aware of his energy usage. After installing the app the user would gain levels by using different features in the app. For instance level three could be gained by logging production of energy, to have tried at least five saving tips or to have shared your power consumption with your friends on Facebook. By using this system the user could set goals like getting a higher lever than his friends.

\subsection{Wishlist}
To achieve even more gamification, the amount of energy saved over the course of using Wattitude should be displayed in the local currency. This allows for another concept, namely that the user can create his own wish list with things he wants to save money for. Wattitude could then show the energy savings of the user as progress towards the value of the items on the list. The app could also display the amount of energy being saved in a given moment in comparison to an earlier moment and how long it would take for the user to save up a given amount of money.  

\subsection{The CoSSMic projects API}
\label{sec:cossmicapi}
Towards the end of the development period the team was made aware of an API that could be used to get data from devices. The team did not have time to implement the system using this API or test this API in any way. This API is meant to work with Emoncms, which is a part of the Open Energy Monitor project as described in section~\ref{sec:openenergymonitor}. The little research done by the team indicates that this API aims to interface the Cossmic project to work with the Emoncms. The team briefly spoke with the people working on the Cossmic project while at Sintef for a demonstration of the Wattitude app. The team learned that they are also using a REST architecture for their system. By modifying the models and databases of the Wattitude app and server, integration with the API should be possible without monumental effort.