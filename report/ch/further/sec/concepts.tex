\section{New concepts}
In this section all ideas for new concepts that the team came up with is described. 

\subsection{Energy Production}
The team has implemented the possibility of creating a device with the category production. This was meant to be a category where all of a residence's energy generating devices could be added. The customer wanted energy production to be in focus, and wanted the possibility of the users to not only monitor their consumption, but their energy production as well. 

It is also possible to implement a way for the user to monitor his usage contra his production, within the usage tab. This way, the user has the possibility to see whether he is capable of selling power to his neighbours. 

\subsection{Gamification}
From the start, the customer uttered a wish of making the act of saving energy around your house, a contest of some sort. Through the concept of gamification. This was discussed within the team and a few different ideas came up. The general thought was that the user should feel that he achieved something, through saving energy and logging it within the app, and therefore would use the app to a greater extent. 

\subsection{Achievements}
One of the ideas that were discussed was the concept of achievement. That the user would get some sort of reward for achieving a saving goal, or conduct a saving tip in his residence. One of the ideas that came up was build on the achievement system that you find in the app "Untappd."~\cite{untappd}
This idea is based on a large base of "badges," or achievements that the user will get after doing something relevant and logging it within the app. 

Another achievement system that was discussed was to integrate different "Wattitude levels" into the app. The idea was to have one level per devotion to saving energy around your house. Say that the user has installed Wattitude, that gets him to the first level because he wants to be aware of his power consumption. And then, the idea moves on to several levels, for instance is level three to produce some energy yourself, to have tried at least five saving tips and to have shared your power consumption with your friends on Facebook. Then, the user can say that his next goal is to reach "Wattitude level 3."

\subsection{Wishlist}
To achieve some level of gamification, whether all amount of energy should also be displayed or translated into the equivalent amount of money. This makes room for another concept, namely that the user can create his own wishlists, with things he want, such as a vacation. And then, the app can display how much energy the user must save to equal the value of the thing he wants. To have a count down, of some sort, was also discussed. For instance, the app can display a status bar which shows how many percent the user has managed to save or has left to save, to have saved the amount of energy equal to the value of the thing he wish for. Or it could display with the amount of energy that is saved in a given moment in comparison to an earlier moment, how long it will take for the user to save up the given amount of money.  