\chapter{Development environment}

The team has in conjunction with the customer decided which technologies to use. These technologies are used as an aid in the development. To solve the task and completing the project, the group utilized the technologies mentioned below.

\section{Development tools and libraries}

The team members have decided to use the integrated development environments Eclipse~\cite{eclipse} and IntelliJ~\cite{intellij}, which are well integrated with Android. 

All the team members already had experience with Eclipse, but some of us also wanted to try out IntelliJ, as it appeared to have more features, such as a faster compiler, better search function and auto completion of code, than Eclipse. However, the remaining part of the team considered the learning curve to switch to IntelliJ to be greater than the advantages, and therefore chose to stick with Eclipse.

\section{Code management and versioning}
\subsection{GitHub}
The team have decided to use Git with GitHub~\cite{github} instead of alternative tools like SVN~\cite{svn} and Mercurial~\cite{mercurial}. 
The team chose this because most of the team members have extensive experience with Git from previous projects 
and the customer requested it as well for overview and easy integration into their existing project. By using 
this setup the team can get better feedback from the customer. 

In addition, GitHub offers very good support for team development 
with advanced functionality like branching and version control to make contributions from several developers easy 
to manage. GitHub also works as an external backup for the project and some of the documentation making data loss less likely.

\subsection{Maven}
Maven is an intelligent project management, construction and application tool. Maven is a tool that can be used to build and manage Java-based projects. The goal of Maven is that developers should understand the complete state of a development project in the shortest time possible.

\subsection{IDEs}
The main IDEs for Java programming are Eclipse, NetBeans and IntelliJ. Most of the team has experience with Eclipse, but IntelliJ has the best support for Android development.
We have decided to do the Android part with IntelliJ and keep the back end part optional. Both IDEs have tools for code quality and code conventions and compatibility between them will only be a minor issue that can be solved with Git.

\subsection{Code convention}
The customer has requested that the team should use Java~\cite{javaconv} and Android's code conventions~\cite{androidconv}.

\section{Documentation}

\subsection{Google Drive}
The team has decided use Google Drive for most of the temporary documentation because it offers an easy way to create 
documents that is accessible for editing and collaboration similtaneously for the entire team. Here, the team will mainly 
keep documents like meeting agendas, information gathering, user inputs,product concepts and other documents 
waiting to get implemented into the final report. 
In addition, Google Drive serves as an external backup, making it unlikely that important documentation will be lost. 


\subsection{\LaTeX}
LaTeX is an advanced typesetting system for document production, widely used in
academic institutions. The system focuses on allowing the author to focus mainly on the content of the document, 
and less on the design and document layout.
The team has chosen LaTeX for the final report instead of other word processing programs like Microsoft Office or 
Google Drive, because LaTeX makes it easier to keep track of references and to maintain the appearance of large
documents. In addition, LaTeX provides the ability to break documents into smaller parts and for several team 
members to similtaneously make changes to the report.


\subsection{JavaDoc}
The team will use JavaDoc~\cite{javadoc} to document the code. The main advantages of using JavaDoc is that it can be 
integrated with the Java code and linked directly with classes and methods making it easier for others to
use the code.

\subsection{Yodiz}
The team's main requirements in a Scrum tool is that it easy to use and effective for project management, 
Git integration, and create charts for documentation purposes. With this tool we should get an overview over 
the projects progress and the team's performance. The team chose Yodiz over other Scrum tools because it provides 
free accounts for educational purposes and it suited the team's requirements. 

\section{Communication}
\subsection{Google groups}
With Google groups we create an arena where we can communicate casually and discuss problems and solutions during 
the development.

\subsection{E-mail}
We have decided to use e-mail as the main way to communicate outside of meetings. E-mail is mainly used to 
share information that is urgent or from external sources.

%\section{Testing}
