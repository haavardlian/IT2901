\chapter{Development environment}

The team has is in discussion with the customer decided which technologies to use. These technologies are used as an aid in development. To solve the task and completing the project, the group used the following technologies.

\section{Development tools and libraries}

The team members have decided to use the  integrated development environments Eclipse and IntelliJ, which is well integrated with Android. 

All the team members already had experience with Eclipse, but some of us also wanted to try out IntelliJ, as it appeared to have more features, such as a faster compiler, better search function and auto completion of code, than Eclipse. However, the remaining part of the team considered the learning curve to switch to IntelliJ to be greater than the advantages and therefore chose to stick to Eclipse.

\section{Code management and versioning}
\subsection{GitHub}
We have decided to use Git with GitHub over alternative tools like SVN and Mercurial. This is because most of us have extensive experience with Git from previous projects and the customer requested it as well for overview and easy integration into their existing project. This way we can get better feedback from the customer. GitHub has very good support for team development with advanced functionality like branching and version control to make contributions from several developers easy to manage. GitHub also works as an external backup for the project and some of the documentation making data loss less likely.

\subsection{Maven}
Maven is an intelligent project management, construction and application tool. Maven is a tool that can be used to build and manage Java-based projects. The goal of Maven is that developers should understand the complete state of a development project in the shortest time possible.

\subsection{IDEs}
The main IDEs for Java programming is Eclipse, NetBeans and IntelliJ. Most of the team has experience with Eclipse, but IntelliJ has the best support for Android development.
We have decided to do the Android part with IntelliJ and keep the back end part optional. Both IDEs have tools for code quality and code conventions and compatibility between them will only be a minor issue that can be solved with git.

\subsection{Code convention}
Java and Android's code conventions.

\section{Documentation}

\subsection{Google Drive}
We will use Google Drive for most of our temporary documentation because it is an easy way to create fast documents that the whole team can collaborate on at the same time. Google Drive also works as an external backup making it unlikely that we will lose any information. We will mainly keep documents like meeting agendas, information gathering, user inputs, product concepts and other documents waiting to get implemented into the final report.

\subsection{LaTeX}
Latex is an advanced typesetting system for document production, widely used in
academic institutions. The system focuses on allowing the author to focus on the content of the document, not the design and how the document looks.
The team has chosen LaTeX, for the final report, over other word processing programs like Microsoft Office or Google Drive because LaTeX makes it easier to keep track of references and to maintain the appearance of large documents as LaTeX has the ability to break documents into smaller parts. It is also possible for several team members to edit the report at different places at the same time.


\subsection{JavaDoc}
The team will use JavaDoc to document the code. The main advantages of using JavaDoc is that it can be integrated with the Java code and linked directly with classes and methods making it easier for others to use the code.

\subsection{Yodiz}
The team's main requirements in a Scrum tool is that it easy to use and effective for project management, Git integration, and create charts for documentation purposes. With this tool we should get an overview over the projects progress and the team's performance. The team chose Yodiz over other Scrum tools because it provides free accounts for educational purposes and it suited the team's demands. 

\section{Communication}
\subsection{Google groups}
With Google groups we create an arena where we can communicate casually and discuss problems and solutions during the development.

\subsection{E-mail}
We have decided to use e-mail as the main way to communicate outside of meetings. E-mail is mainly used to share information that is urgent or from external sources.

%\section{Testing}
