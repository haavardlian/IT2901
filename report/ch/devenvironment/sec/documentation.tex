\section{Documentation}

\subsection{Google Drive}
The team has decided use \gls{gdrive} for most of the temporary documentation because it offers an easy way to create documents that is accessible for editing and collaboration simultaneously for the entire team. Here, the team will mainly 
keep documents like meeting agendas, information gathering, input from the customer and the supervisor, product concepts and other documents waiting to get put into the final report. 

In addition, Google Drive serves as an external backup, making it highly unlikely that important documentation will be lost. 


\subsection{\LaTeX}
LaTeX~\cite{latex} is an advanced typesetting system for document production, widely used in
academic institutions. The system focuses on allowing the author to focus mainly on the content of the document, and less on the design and document layout.

The team has chosen LaTeX for the final report instead of other word processing programs like Microsoft Office or Google Drive, because LaTeX makes it easier to keep track of references and to maintain the appearance of large
documents. 

In addition, LaTeX provides the ability to break documents into smaller parts and for several team members to simultaneously make changes to the report.

\subsubsection{Maintaining the quality of the report}
Our biggest goals throughout this project was to produce an application both the team and the customer would be greatly satisfied with, but also to produce a report of high quality that documents the project's process. 

To strive for such a report, the team used some tools that proved to be very helpful during the production of the report. These tools are described in the subsequent sections.

\paragraph{\textbackslash todonotes}
Feedback is key when creating a product. The \textbackslash todonotes~\cite{todo} package allowed the team to comment on paragraphs or formulations we wanted to rephrase, whether there was missing an illustration and also gave us a list of all the things we had to do, making it easy to get an overview of the remaining tasks regarding the report.

\paragraph{Spellchecking: aspell}
Although manual proofreading cannot be avoided, it is advantageous to have a tool to perform automatic spellchecking. Aspell~\cite{aspell} is such a tool.

\paragraph{References and glossary}
To keep track of the citations in the bibliography in the report, we used a LaTeX-package called BibTex~\cite{bibtex}, and for our glossaries, a package called glossaries~\cite{glossaries}.


\subsection{JavaDoc}
The team will use JavaDoc~\cite{javadoc} to document the code. The main advantages of using JavaDoc is that it can be integrated with the Java code and linked directly with classes and methods making it easier for others to use the code.

\subsection{Yodiz}
The team's main requirements in a Scrum tool is that it easy to use and effective for project management, has Git integration, and that can create charts for documentation purposes. With this tool the team should be able to get an overview over the projects progress and the team's performance. 

The team chose Yodiz~\cite{yodiz} over other Scrum tools because it provides 
free accounts for educational purposes and it suited the team's requirements. 