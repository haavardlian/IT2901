\section{Code management and versioning}
\subsection{GitHub}
The team have decided to use Git with GitHub~\cite{github} instead of alternative tools like SVN~\cite{svn} and Mercurial~\cite{mercurial}. 
The team chose GitHub because most of the team members have extensive experience with Git from previous projects. In addition, it was requested by the customer. 

By also taking the customer in this matter into consideration, we would be more able to give the customer a better overview of the project and also to easily integrate our application into the customer's existing project. As a consequence, both the team and the customer would be updated on the last version of the project and the team would be more likely to receive relevant feedback. 

In addition, GitHub offers very good support for team development 
with advanced functionality like branching and version control to make contributions from several developers easy 
to manage. GitHub also works as an external backup for the project and some of the documentation making data loss less likely.

\subsection{Development in Android}
For the development in Android, the team has decided to use Android Studio~\cite{android-studio}, which is an integrated development environment based on IntelliJ~\cite{intellij}, that is specifically designed for development on the Android platform.

Despite the fact that some the team members already had experience with Android Studio's counterpart Eclipse ADT~\cite{eclipseadt} and few of us had any experience with Android Studio, the team wanted to give it a try, as it appeared to have more features, such as a faster compiler, better search function and auto completion of code, than Eclipse ADT. 

As the project has progressed, it has become clear that the learning curve has not been as steep as expected. Neither have we encountered any problems that we did not think also would have been present if we had chosen Eclipse ADT.

\subsection{Integrated Development Environments (IDEs)}
The main IDEs for Java programming are Eclipse~\cite{eclipse}, NetBeans~\cite{netbeans} and IntelliJ. Most of the team members has previous experience with Eclipse for both Java and Android development, but IntelliJ has the best support for Android development, which is further explained in the section above.

The team has decided to keep the back-end part optional. Both IDEs have tools for checking the code quality and code conventions, and compatibility between them will presumably be a minor issue that may be resolved with Git.

\subsection{Maven}
Maven~\cite{maven} is an intelligent project management, construction and application tool. Maven is a tool that can be used to build and manage Java-based projects. The goal of Maven is that developers should understand the complete state of a development project in the shortest amount of time possible.

\subsection{Gradle}
Gradle~\cite{gradle} is in many ways similar to Maven. It is used to build and manage the Android part of the project. The reason we use gradle instead of Maven for the entire project is because gradle is well integrated in the Android community and most of the libraries needed for the android application is supported through gradle.

\subsection{Code convention}
The customer has requested that the team should use Java~\cite{javaconv} and Android's code conventions~\cite{androidconv}.

\todo[inline]{should we say something about gradle?}
