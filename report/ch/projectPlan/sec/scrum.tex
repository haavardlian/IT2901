\section{Scrum}

% BRUK TEAM ISTEDENFOR GROUP

The assignment at hand will demand innovation and frequent changes underways.
What the group needs is an agile development framework to follow. First and foremost it must also be a development framework that is known to the group, 
so no time will be spent trying to learn a new process.
The standard choice for this is to follow the Scrum model. All the members of the group have previous experience with this from the course IT1901.

The Scrum model is an iterative and incremental agile software development framework.
The main principle is that a small group is focusing on reaching a common goal.

The scrum process consists of intervals of sprints that are 1 to 4 weeks long. Each sprint has three important parts.
The first part is the sprint planning meeting, then begins the daily scrum meetings and the process is ended with a so called end meeting.

\subsection{Sprint planning}
The objective of the sprint planning is to find out what work to be done and prepare a
sprint backlog that consists of the tasks to be done, a so called user story, and how much time the team thinks it will take to complete it. A normal strategy to use for time planning is
planning poker. For each user story everyone will decide for their own how many units of time they think it will take. Units of time can be in work day, hours, shirt sizes or whatever the group sees fit. 
Then everyone presents their guess. If the group agrees on their guesses, then the user story gets that time. If not, the group can discuss their choices and come to a conclusion everyone agrees on.
After this meeting the team should have a well prepared strategy for the sprint.

\subsection{Daily meetings}
Daily meetings happen at the start of the day. It should only be around 15 minutes. The point of this meeting is that everyone tells what they have done and if something has gone wrong.
Maybe the time analysis for a user story was wrong and demands more though and replanning. The result of having this meeting is that problem that arises are dealt with quickly.

\subsection{End meeting}
The end meeting happens at the end of a sprint. The meeting consists of a sprint review and retrospective discussion for the last sprint.
Thus progress is reviewed, and accumulated lessons from the sprint can be taken in account for the next sprint.
It's also typical to have customer meetings to show what has been done so the customer can give feedback to wheter the product is developing to a satisfactory product.

\subsection{Our process}
We have chosen to have sprint that are two weeks long. Having daily meetings is not possible with our schedule,
since this course demands 20 hours each week and the team members is taking different courses besides this one.
The group has come to an conclusion to have two meetings each week that begins with a daily meeting.

We have also decided to use a Scrum tool to help us in the process.
What we want is a tool that will give us automatic sprint backlogs, time measurements for each user story, graphs, and an interactive scrum board.

Out team has previously had experience with Scrum tools like icescrum. This experience was not a jolly one.
After having reviewed lots of different cloud based Scrum tools, we ended up using Yodiz (http://www.yodiz.com).

The development tasks will be divided in epics, user stories and tasks. Epics are user stories that are more general, and demands lots of hours to complete. Each epic consists
of all the user stories that explain the usage of the epic in more detail. The user stories will have time estimates that are discussed in the group under sprint planning meetings.
The sprint backlog will consist of the user stories that the group has decided to tackle.

A user story is given smaller tasks that are the different technical aspects of the story that needs to be done.

During a sprint the progress is handled in a scrum board. The scrum board has four columns. One for the user stories, new, in progress, and done.
All the tasks for the user story are in the new column, then dragged appropriately to the correct column during the sprint, and updated with time usage.
