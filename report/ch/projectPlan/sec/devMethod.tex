\section{Development method}
Will you work iteratively and/or incrementally, will you make a 
mock-up or prototype; phases and iteration



%\subsection{Resources}
\subsection{Risk analysis}
As part of our project planning, we have outlined potential risks to the progress of our work. A risk
is defined as an unwanted event that affects the process. We acknowledge the possibility of challenges and problems arising during the work, such as technical problems, human error or personal problems. These are risks for both individuals and the group as a whole. Effects of these problems include delays, conflict and anything that will slow the progress, and ultimately lead to failure to  meet deadlines. One of the most prevalent and dangerous risks is loss of motivation from one or more members of the group. Our assessment of the risks to this project is included in attachment E. The risk elements are sorted by their importance. Importance is calculated with two factors in mind; the calculated probability of the event actually occurring and the effect the event will have on the project. 
In the table given below, we analyze the elements we consider risks to our project. Likelihood (L) and effect (E) is measured on a scale from 1 to 9. For likelihood, a 9 is very likely and a 1 is very unlikely. For effect, a 9 is devastating and a 1 is not much effect. The table is sorted from high to low importance (I). Importance is the product of probability and effect. 




\begin{table}[H]
\begin{tabular}{|p{4cm}|p{0.2cm}|p{0.2cm}|l|p{4.8cm}|p{4.8cm}|}
\hline
Description &L& E& I& Preventive action & Remedial action \\\hline
Customer requirements exceeds predicted amount of time&&&& Continiously revise the workload and prioritize &  \\\hline
Underestimation of workload & &&& Continiously revise workload and how much time is left, and prioritize & Reestimate continiously\\\hline
Data loss&&&& Use version control system &Restore data from previous versions\\\hline
Conflict in group& & & & Have in mind that people often misunderstand each other and have an open dialog to solve problem. & Contact supervisor for help.\\\hline
Communication failure &&&& Make e-mail-list and exchange contact information. Be explicit. & Check e-mail and phones multiple times a day.\\\hline
Unbalanced workload&&&& Coordinate with entire team and log hours & Reallocate tasks\\\hline
Issues with software or tools&&&& Choose software the team members are familiar with & Hold workshops and view tutorials\\\hline
Eclipse and IntelliJ is imcompatible&&&&&\\\hline
External services unavailable &&&&&\\\hline
Team member unavailable &&&& Good infrastructure for communication and progress reports & Keep in touch with unavailable team member or redistribute tasks\\\hline
Team member drops out &&&& Keep track of the effort and the hour's log public & Contact supervisor and redistribute tasks \\\hline
Illness &&&&Multiple team members work on the same task& Allocate sick member's task to remaining members\\\hline
Supervisor unavailable&&&& Keep regular contact with supervisor & Discuss problem internally and contact department\\\hline
Customer unavailable&&&& Keep regular contact with customer & Discuss problem within team and contact supervisor\\\hline
\end{tabular}
\end{table}


%\subsection{Work Breakdown Structure}
\subsection{Architecture}
