\section{Development method}
Will you work iteratively and/or incrementally, will you make a 
mock-up or prototype; phases and iteration



%\subsection{Resources}
\subsection{Risk analysis}
Gruppen har diskutert ulike hendelser som kan påvirke progresjonen i prosjektarbeidet. En risiko
kan defineres som en mulighet for at noe uønsket skjer. Vi har sett for oss at det kan dukke
opp mange utfordringer og ulike risikoer i løpet av prosjektperioden, som for eksempel tekniske
utfordringer eller menneskeskapte og personlige problemer. Dette kan gjelde både for enkeltpersoner
og hele gruppen. Problemene kan skape forsinkelser, konflikter, mangel på progresjon og føre til at
prosjektgruppen ikke rekker de ulike innleveringsfristene. En annen risiko som er avgjørende for
resultatet, er hvorvidt alle gruppemedlemmene er motivert til å fullføre prosjektet.
I vedlegg E har gruppen vurdert flere scenarioer som kan hindre arbeidet i større eller mindre
grad. Elementene er sortert etter betydning for prosjektet. Betydningen er beregnet på bakgrunn av
sannsynligheten for at hendelsen inntreffer, og påvirkningen hendelsen vil ha for prosjektarbeidet.

I tabellen under tar vi for oss en analyse av risikoelementer som kan oppstå i løpet av prosjektet.
Sannsynlighet(S) og innvirkning(I) måles på en rangering fra 1-9, hvor 9 er høyst sannsynlig,
og 1 er svært usannsynlig. Innvirkning(I) på en skala fra 1-9, hvor 9 er høyst sannsynlig, og 1 er
svært usannsynlig. Tabellen er sortert fra høy til lav betydning(B), produktet av sannsynlighet og
innvirkning.


\begin{table}[H]
\begin{tabular}{|p{4cm}|p{0.2cm}|p{0.2cm}|l|p{4.8cm}|p{4.8cm}|}
\hline
Description &L& E& I& Preventive action & Remedial action \\\hline
Customer requirements exceeds predicted amount of time&&&& Continiously revise the workload and prioritize &  \\\hline
Underestimation of workload & &&& Continiously revise workload and how much time is left, and prioritize & Reestimate continiously\\\hline
Data loss&&&& Use version control system &Restore data from previous versions\\\hline
Conflict in group& & & & Have in mind that people often misunderstand each other and have an open dialog to solve problem. & Contact supervisor for help.\\\hline
Communication failure &&&& Make e-mail-list and exchange contact information. Be explicit. & Check e-mail and phones multiple times a day.\\\hline
Unbalanced workload&&&& Coordinate with entire team and log hours & Reallocate tasks\\\hline
Issues with software or tools&&&& Choose software the team members are familiar with & Hold workshops and view tutorials\\\hline
Eclipse and IntelliJ is imcompatible&&&&&\\\hline
External services unavailable &&&&&\\\hline
Team member unavailable &&&& Good infrastructure for communication and progress reports & Keep in touch with unavailable team member or redistribute tasks\\\hline
Team member drops out &&&& Keep track of the effort and the hour's log public & Contact supervisor and redistribute tasks \\\hline
Illness &&&&Multiple team members work on the same task& Allocate sick member's task to remaining members\\\hline
Supervisor unavailable&&&& Keep regular contact with supervisor & Discuss problem internally and contact department\\\hline
Customer unavailable&&&& Keep regular contact with customer & Discuss problem within team and contact supervisor\\\hline
\end{tabular}
\end{table}


%\subsection{Work Breakdown Structure}
\subsection{Architecture}
