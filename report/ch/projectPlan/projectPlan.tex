\chapter{Project schedule}
\input ch/projectPlan/sec/teamorganization.tex
\input ch/projectPlan/sec/mainresponsibility.tex
\input ch/projectPlan/sec/planning.tex



\section{Development method}
Will you work iteratively and/or incrementally, will you make a 
mock-up or prototype; phases and iteration



%\subsection{Resources}
\subsection{Risk analysis}
Gruppen har diskutert ulike hendelser som kan påvirke progresjonen i prosjektarbeidet. En risiko
kan defineres som en mulighet for at noe uønsket skjer. Vi har sett for oss at det kan dukke
opp mange utfordringer og ulike risikoer i løpet av prosjektperioden, som for eksempel tekniske
utfordringer eller menneskeskapte og personlige problemer. Dette kan gjelde både for enkeltpersoner
og hele gruppen. Problemene kan skape forsinkelser, konflikter, mangel på progresjon og føre til at
prosjektgruppen ikke rekker de ulike innleveringsfristene. En annen risiko som er avgjørende for
resultatet, er hvorvidt alle gruppemedlemmene er motivert til å fullføre prosjektet.
I vedlegg E har gruppen vurdert flere scenarioer som kan hindre arbeidet i større eller mindre
grad. Elementene er sortert etter betydning for prosjektet. Betydningen er beregnet på bakgrunn av
sannsynligheten for at hendelsen inntreffer, og påvirkningen hendelsen vil ha for prosjektarbeidet.

I tabellen under tar vi for oss en analyse av risikoelementer som kan oppstå i løpet av prosjektet.
Sannsynlighet(S) og innvirkning(I) måles på en rangering fra 1-9, hvor 9 er høyst sannsynlig,
og 1 er svært usannsynlig. Innvirkning(I) på en skala fra 1-9, hvor 9 er høyst sannsynlig, og 1 er
svært usannsynlig. Tabellen er sortert fra høy til lav betydning(B), produktet av sannsynlighet og
innvirkning.


\begin{table}[H]
\begin{tabular}{|p{4cm}|l|l|l|p{4.5cm}|p{4.5cm}|}
\hline
Beskrivelse  &S& I& B& Forberedende tiltak & Utbedringstiltak \\\hline
Too much to do&&&&& Kontinuerlig vurdere hvor mye \\\hline
Divergent time estimation & &&& Keep track of amount of remaining work and how much time is left & Reestimate\\\hline
Data loss&&&& Take daily backups &Restore data from previous backups\\\hline
Conflict in group& 6& 4& 24& &Ta opp problemene og ha en åpen diskusjon. Legge det tilside eller evt. spørre studentassistent eller høyere instanser om hjelp til å løse konflikten\\\hline
Communication failure &&&&&  \\\hline
Feil fordeling av oppgaver&&&&&\\\hline
Different professional skills and background&&&&&\\\hline
Issues with software &&&&&\\\hline
Team member not available &&&&&\\\hline
Team member busy &&&&&\\\hline
Team member drops out &&&&&\\\hline
Short-term illness &&&&&\\\hline
Long-term illness&&&&&\\\hline
Supervisor not available&&&&&\\\hline

\end{tabular}
\end{table}


%\subsection{Work Breakdown Structure}
\subsection{Architecture}
