
\chapter{Project schedule}

\section{Team organization}
In order to exploit the team's resources in the best way possible, all the team members has summed up their background knowledge and which role they wish to have in the project. The team has organized its structure the based on this information.

\subsection{The team members}
The team consists of six persons, all in their final year of a bachelor in Informatics. All the members have previous experience on working in teams on educational projects.

\subsubsection{Beate Baier Biribakken}
Beate have previously worked at Student-Media AS and Sportradar AS as web developer. From these experiences, she gained knowledge about Linux and web development, such as PHP, JavaScript, HTML and CSS. She also has some experience with Scrum, Java and Android.

\subsubsection{Tor-Håkon Bonsaksen}
Tor Håkon has..
\subsubsection{Lars Erik Græsdal-Knutrud}
 Lars Erik has..
\subsubsection{Per Øyvind Kanestrøm}
Per Øyvind has..

\subsubsection{Håvard	Holmboe	Lian}
Håvard has ..
\subsubsection{Pia	Karlsen	Lindkjølen}
Pia has ..

\section{Main responsibilities}
\begin{table}[H]
\begin{tabular}{l|p{7cm}}
\textbf{Role} & \textbf{Description}\\\hline
Project leader: Pia & Keeping team updated and monitor the project's status\\\hline
Deputy project leader: Lars Erik & Fill in whenever the project leader is incapable to perform all duties\\\hline
Scrum-master: Per Øyvind & Make sure that Scrum-process goes as smooth as possible, that the team provides necessary documentation and keep track of the project process\\\hline
Customer relations: Pia & All customer communication mainly goes through this person.\\\hline
Development: Tor-Håkon & Keep track of the technological development progress, make sure it is going by schedule and take necessary action \\\hline
Report: Beate & Monitor the report's progress, spell check and review content.\\\hline
Testing: Håvard & Make sure that the code is properly tested during the development process to detect possible errors, deficiencies and bugs and take the necessary action \\\hline
Secretary: circulates & Take note of important information during meetings within the team and with the customer.
\end{tabular}
\end{table}

\section{Development method}
Will you work iteratively and/or incrementally, will you make a 
mock-up or prototype; phases and iteration

\section{Planning}
\subsection{Time schedule}
The assignment and team members were assigned January 20. The initial meeting with the customer took place January 24, where the assignment was presented and discussed. Oral presentation of the project will be performed March 19 and the final deadline for submission of the project is set to May 30.

%mention trip to China
 
 Table~\ref{tab:availHours} lists the project's available hours and is based on that all of the team members spends twenty hours on the project every week.

\begin{table}[H]
\centering
\rowcolors{1}{darkgray}{lightgray}
\begin{tabular}{llll}
\textbf{Period} & \textbf{Dates} & \textbf{Days} & \textbf{Hours}\\
Introduction& - &5  & 120 \\
Sprint 1& - & 5  & 120 \\
Sprint 2 & - &10  & 240 \\
Sprint 3 & - &10 & 240 \\
Sprint 4 & -&10  &240 \\
Sprint 5 & - &10&  240 \\
Sprint 6 &0 - &10  &240 \\
\textbf{Total}&& \textbf{-}&  \textbf{-}
\end{tabular}
\caption{Available hours}
\label{tab:availHours}
\end{table}

\subsubsection{Milestones}
%\subsection{Resources}
%\subsection{Risk analysis}
%\subsection{Work Breakdown Structure}
\subsection{Architecture}