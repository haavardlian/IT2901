\section{Sprint 1}
goal for sprint
execution
implementation
testing
design
project administration
result

% kutt her til beskrivelse av valg av scrum-verktøy. Bør legges inn under riktig sprint
\subsection{Election process of Scrum tools}
\label{sec:scrumtools}
The team has previous experience with a Scrum tool called iceScrum~\cite{icescrum}. After a discussion, the team came to the conclusion 
that iceScrum was not to be used again and that an alternative was needed.
After having reviewing lots of different cloud based Scrum tools, the team ended up with Yodiz~\cite{yodiz}.

The project will be divided into epics, user stories and tasks. Epics are user stories that are more general, 
and demand at least twelve hours of work to complete. Each epic consists
of all the user stories that explain the usage of the epic in more detail.
The user stories will have time estimates discussed during sprint planning meetings.
The sprint backlog will consist of the user stories that the team has decided to handle.

A user story is given smaller tasks that are the different technical aspects of the story that needs to be done.

During a sprint the progress is handled in a Scrum board. The Scrum board has four columns. One for the user stories, new tasks, tasks in progress, and completed tasks.
All the tasks for the user story are in the new column and on the same row as it's user story, then dragged appropriately to the correct column during the sprint, and updated with time usage.
