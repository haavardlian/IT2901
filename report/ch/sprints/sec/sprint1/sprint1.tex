\section{Sprint 1}
%goal for sprint
%execution of sprint
%development
%testing
%design
%project administration
%result + lessons learned

\subsection{Goals for the sprint}
The team's first sprint had several diverse goals, which is common for the start up phase of any project. The goals can be divided roughly into four different categories: Project management, product specification, early product design and documentation.

\subsubsection{Project management}
\todo[inline]{The first sprint was mainly about organizing how the team wanted to work, what the team wanted work with, and discussing how to reach these goals.}

The first sprint contained a lot of management and planning. The team mapped its resources and tried to get an overview of the task at hand. The team did a clarification of the expectations to the project. All team members agreed to aiming for a \todo[inline]{a good grade} and at the same time learn more about both group work and android development. Suitable working hours for all group member was also discussed and found. Additionally, tools and technologies had to be chosen for the initial superficial system architecture. The rest of the work in management consisted of allocating responsibilities, drawing up a rough project plan and coming up with some possible concepts for the system.

\subsubsection{Product specification}
The functions and design system to be developed turned out to be very open for interpretation. The customer was a bit vague in describing the product at first, but he did have some concrete requirements: The product should include an Android app, and the app should raise awareness on power consumption in private homes. The customer also communicated that the app should be integrated with social media somehow. We turned our ideas and the customer's wishes into a draft of the requirements specification that the customer later approved.

\subsubsection{Product design}
With a specification and some tools chosen beforehand, the team started looking for open source technologies that could be utilized in the development of the product. After sketching the architecture of the entire system, including the app and supporting systems, the team decided on a set of libraries and tools as aid in our future development.

\subsubsection{Documentation}
To make sure that documentation was not neglected early on, the team started working on it immediately. A report template was created using LaTeX so that the team easily could update the documentation continuously.

\todo[inline]{Add subsection about the execution of this sprint, how we want to reach our goals}

\todo[inline]{Add this under subsection of Administration}
\subsection{Selection of tools}
As part of the teams project management, the team needed to agree on what tool sets to use during the project.

\subsubsection{Election process of Scrum tools}
\label{sec:scrumtools}
The team started by researching what tools that satisfied our Scrum criterias~\ref{sec:sprintplanning}. Some of the tools that was suggested was iceScrum~\cite{icescrum}, Yodiz~\cite{yodiz}, ScrumDo and Jira.

The team has previous experience with iceScrum. After a discussion, the team came to the conclusion that iceScrum was not to be used again and that an alternative was needed. The reason why this was decided is that iceScrum must be deployed on a server, \todo[inline]{and it is restrictive in how to use it.}

Some of the tools was professional but optimized for enterprise use. Others lacked professionalism. After having reviewed the different cloud-based Scrum tools, the team ended up with Yodiz as the best alternative. The winning arguments was a good sprint board, the ease of use, many positive reviews and the pricing, which was free for educational purposes.

The project was divided into epics, user stories and tasks. Epics are user stories that are more general, and that demand at least twelve hours of work to be completed. Each epic consists of all the user stories that explain the usage of the epic in more detail.
The user stories will have time estimates discussed during sprint planning meetings.
The sprint backlog will consist of the user stories that the team has decided to handle.

A user story is given smaller tasks that are the different technical aspects of the story that needs to be done.

During a sprint the progress is handled in a Scrum board. The Scrum board has four columns. One for the user stories, new tasks, tasks in progress, and completed tasks.
All the tasks for the user story are in the new column and on the same row as it's user story, then dragged appropriately to the correct column during the sprint, and updated with time usage.

\subsubsection{Choice of IDE}
For android development Android Studio from Google was chosen. I has very good integration with GUI development i Android, and also it builds on the IntelliJ IDE of which several of the team members had prior knowledge. As Gradle was being used in building the project, it was left open for team members to chose IDE when working on the server implementation of the system.

\subsection{Results}
The team made several key decisions during sprint 1 regarding project methodology, tools and architecture. The customer was very pleased with the teams ideas and concepts for the app, and approved of the temporary specification. The team members deemed sprint 1 very successful.
