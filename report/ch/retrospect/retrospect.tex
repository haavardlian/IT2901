\chapter{Retrospect}
In this chapter, the objective is to evaluate all of the choices the team has made during this course. This includes the time spent, the entire development process - from choosing IDEs to testing the system and the project management. 


\input ch/retrospect/sec/timespent.tex


\subsection{The most influential risks}
As explained in the section ~\ref{sec:risk}, the team analyzed the possible risks that might occur during the time scope of the project. The risk analysis in table ~\ref{tab:risktable} gives a certain idea about how much of an impact the team expected a risk to have, but says nothing about which risks that actually occurred and that had a big impact on the project. This section gives a short description of the most influential risks the team experienced throughout the project.

\subsubsection{Underestimation of workload}
A prevalent risk to any project is the failure to acknowledge the cost of tasks in the project and the improper allocation of resources that follows such an error. This can also lead to a false sense of what a team can achieve in a given time frame. 

The team experienced this risk to have the biggest negative impact on our workflow. Even though it did not have much effect on the sprint end result, it was the most occurring risk, which added up to be very time consuming. The main reason for this issue was the lack of experience with many of the tasks, but also that we initially did not consider who the tasks were assigned to, which turned out to play an important role in the equation. How the team resolved this issue is further explained in section~\ref{sec:assignEst}.

\subsubsection{Issues with software or tools}
Any team of software developers usually rely on a set of tools and software to help them develop and document a system. Unavailability, lack of experience or improper use of such tools can lead to setbacks in progress.

The team experienced this initially both with the Scrum tool, as explained in section~\ref{sec:improperScrum}, and when setting up the Android development environment.

\subsubsection{Unbalanced workload}
Failure to estimate the cost of tasks, or failure to allocate them wisely, can leave some team members grasping for more work while others are bogged down with either too heavy tasks or a much greater workload.

The team did experience this, but considered it to be more of an opportunity to improve the use of resources and streamline the workflow rather than being a high impact issue. Further details about this issue may be found in section~\ref{sec:unbalancedWorkload}.


\section{Development process}
SECTION NOT FINISHED. For this process, the team chose to use scrum together with extreme programming. (Did we add extreme programming after a little while? Should this be mentioned?) Even though all of the team members have previous experience with scrum, there were some disagreements. Because scrum is very flexible so members have used it differently? If we added extreme programming after a while we have to say why here. 

OMFORMULERES: Each team member spent a significant amount of time getting up to date with the Android platform and the front-end architecture of Android apps. In retrospect, more time should have been spent getting all group members up to date and practicing Android programming together in the beginning of the project. One of the team members had some experience with Android before and ended up spending a lot of time teaching and explaining to all of the other team members. If this had been organized in a way, for instance by holding a course in the beginning, some of the time spent on this could have been saved. 

\subsection{Technical choices}
The team agreed on using Android studio from the beginning, after discussing whether to use Android studio or eclipse with an Android plugin. In the beginning there were some problems with Android Studio. Some team members had trouble actually running the app. There was a problem with Android Studio not finding the correct SDK. (<-IS THIS CORRECT?) This took some time to solve, but after it was solved the team had no more trouble with is. However, since Android Studio is a relatively new IDE, is still has some start trouble and updates comes out regularly. This meant that all of the group members had to use at least two or three minutes every time a new update came out. 

We changed the architecture during the process, we should write about that and why here, do we think we chose the best one? We chose to create a provider, do we feel it was worth all the extra time it took? 

\subsubsection{Choice of Android IDE}
Despite the fact that some the team members already had experience with Android Studio's counterpart Eclipse ADT~\cite{eclipseadt}, and few of us had any experience with Android Studio, the team wanted to give it a try, as it appeared to have more features, such as a faster compiler, better search function and auto completion of code, than Eclipse ADT. 

As the project has progressed, it has become clear that the learning curve has not been as steep as expected. Neither have we encountered any problems that we did not think also would have been present if we had chosen to use Eclipse ADT.

\subsubsection{Choice of GitHub for code management}
By taking the customer's opinion in this matter into consideration, we were be more able to give the customer a better overview of the project and also to easily integrate our app into the customer's existing project. As a result, both the team and the customer were updated on the last version of the project and the team was more likely to receive relevant feedback. 

\section{Project management}
SECTION NOT FINISHED. As mentioned earlier, the team chose scrum. Yodiz, we now wish we had chose another scrum tool. After a while we added a person responsible for some serving on our work days. Also responsible for making sure we took brakes. 
The team introduced punishments for being to late for meetings and not doing all of your assignments. This was used on a team building with some food. Did this work optimally?
\subsection{Working with the customer}
SECTION NOT FINISHED. The team tried to have meetings with the customer once a week. This worked well and the team felt that it was a good thing that the customer participated to such an extent. Write something about the problems with the customer saying he is going to do something but did not and how we solved the problems around this?

\subsection{Team organization}
SECTION NOT FINISHED.
In the beginning of the project, the team distributed roles with different responsibilities. A project leader, a deputy project leader, a scrum master, a development responsible, a test responsible and a report responsible. Almost halfway through the project it became clear that the scrum master was also the one with the most Android experience. This resulted in a lot of extra work on the scrum master, while the deputy project leader had almost no work at all. Therefore, the team reflected over the project roles and the distribution and this resulted in a change of roles. The deputy project leader became the scrum master to distribute the resources in the best possible way. After this change the teams best Android resource could focus entirely on developing and helping others. The team was very satisfied with this decision and the resource distribution worked great after the change in roles. 

Was the responsibilities of the roles good?
