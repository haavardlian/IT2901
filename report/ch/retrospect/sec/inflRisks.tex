\section{The most influential risks}
As explained in section~\ref{sec:risk}, the team analyzed the possible risks that might occur during the time scope of the project. The risk analysis in table~\ref{tab:risktable} gives a certain idea on how much of an impact the team expected a risk to have, but says nothing about which risks that actually occurred and that had a big impact on the project. This section gives a short description of the most influential risks the team experienced throughout the project.

\subsection{Underestimation of workload}
A prevalent risk to any project is the failure to acknowledge the cost of tasks in the project and the improper allocation of resources that follows such an error. This can also lead to a false sense of what a team can achieve in a given time frame. 

The team experienced this risk to have the biggest negative impact on our workflow. Even though it did not have much effect on the sprint end result, it was the most occurring risk, which added up to be very time consuming. The main reason for this issue was the lack of experience with many of the tasks, but also that we initially did not consider who the tasks were assigned to, which turned out to play an important role in the equation. 

\todo{explain how the team resolved this issue?}

\subsection{Issues with software or tools}
Any team of software developers usually rely on a set of tools and software to help them develop and document a system. Unavailability, lack of experience or improper use of such tools can lead to setbacks in progress.

The team experienced this both with the Scrum tool, as described in section~\ref{sec:improperScrum}, and when setting up the Android development environment.

\subsection{Unbalanced workload}
Failure to estimate the cost of tasks, or failure to allocate them wisely, can leave some team members grasping for more work while others are bogged down with either too heavy tasks or a much greater workload.

The team did experience this, but considered it to be more of an opportunity to improve the use of resources and streamline the workflow rather than being a high impact issue. Further details about this issue may be found in section~\ref{sec:unbalancedWorkload}.
