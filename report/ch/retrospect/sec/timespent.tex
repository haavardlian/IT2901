\section{Allocation of time and resources}
This section presents an overview of the time and resource allocation in the project.

\subsection{Time allocation}
At the beginning of the project, the team did an estimation on how much time we would spend on each part of the project. In figure~\ref{fig:piechart} we show an overview on how much time we actually spent.

\begin{figure}[H]
\centering
\includegraphics[width=0.6\textwidth, clip, trim=4cm 2cm 4cm 1cm]{ch/retrospect/fig/timePie.png}
\caption{Pie chart of time spent on different parts of the project.}
\label{fig:piechart}
\end{figure}

When comparing the actual time spent with the estimated time from table~\ref{tab:timeEstWP}, we see that the team spent more time on \todo{add more here when illustrations are ready}

\begin{figure}[H]
\centering
\includegraphics[width=0.7\textwidth, clip, trim=1.1cm 0.5cm 1.2cm 1cm]{ch/retrospect/fig/release.png}
\caption{Project release burn down chart}
\label{fig:release}
\end{figure}

\noindent In the release burn down chart, displayed in figure~\ref{fig:release}, we compare the estimated project progress with the progress we actually made. As the graph shows, the team worked less than the estimated number of hours. There are several reasons for this, including unplanned absence of team members, underestimation of tasks and too heavy workload due to deliveries in other subjects. Another large factor in the discrepancy between the planned hours and the actual time spent is the manner of which work hours are logged. The hours logged in Yodiz, which are shown in the graphs of this chapter, only represent how long tasks took to complete. The logging done in Yodiz is used primarily to estimate user stories and tasks, and not to log precise work hours. This gives a slight deviation in hours spent and hours logged, which over the course has amounted to a notable amount. Fortunately, the team had deliberately overestimated the number of work hours to compensate for lost time due to the school trip to China. This action resulted in that the team still was able to fulfil the course requirement and complete the assignment within the project's time scope.

\todo{oppdatere release og pie charten når sprint 7 og 8 er ferdig skrevet}

\subsection{Resource allocation}
As described in section~\ref{sec:availResources}, the team had several resources available, including a supervisor, the customer and, of course, the other team members. The team had internal meetings at least two times a week, and also an eight hour work session once a week. In addition, we had weekly meetings with the customer and meetings with the supervisor every other week.

To have meetings at such regular times has been of great value for the team. It saved a lot of time to have pre-booked meeting rooms, and it was also easier for the parties involved in the project to prepare for the meetings.

On the meetings with the supervisor, we were given valuable input and feedback regarding customer relations, and the report content and structure. 
Communicating with the customer on a regular basis has reduced the probability of misunderstandings arising and increased the likelihood that the team would deliver a product the customer would be greatly satisfied with. 

Some of our greatest decisions in this project has been to actually have fixed meeting times, assuring that the entire team would be kept updated and involved. On these internal meetings, the team planned and distributed tasks among the team members, and discussed occurring issues. We also had assigned each team member an area of responsibility, so that no important parts of the project would be neglected, but also as a way to keep the motivation high and further involve the team members.






\todo{Jeg vet ikke om vi trenger å ha med dette(i todo-boksen). Hvis du føler noe av dette bør stå i denne seksjonen, plz add.
Was all the time spent with the supervisor and the customer optimal? Did we need all the resources given to us? Student assistant, group meetings with the class. Did we get the most out of our own group meetings or could we have done something different?
In the beginning, the team found the supervisors role somewhat confusing. After one or two meeting, and explicit asking the supervisor it became more clear to us that the supervisor was an overall support with focus on our report. And how we evaluated the risks throughout the project.
}