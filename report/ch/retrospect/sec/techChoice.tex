\newpage
\section{Technical choices}
Even though the team aimed to make informed decisions before the development started, some changes had to be made during the project. This section presents an evaluation of these decisions.

\subsection{Android development}
When it comes to the Android development, a great deal of the decisions made turned out to be reasonable. Few of the team members had previous experience with development in Android.

Each team member spent a significant amount of time getting up to speed with the Android platform and the front-end architecture of Android apps. In retrospect, more time should have been spent getting all team members up to date and practicing Android programming together in the beginning of the project. One of the team members had some experience with Android development and ended up spending a lot of time teaching and explaining to the other team members. This could have been organized in a way, for instance by holding a workshop in the beginning. Learning Android development properly from the start could have saved a lot of time.

\subsubsection{Development tools}
The team agreed on using Android Studio after a brief discussion on whether to use Android Studio or Eclipse with ADT.
 
At the start of the project, the team experienced some issues with Android Studio - some of the team members had trouble compiling the app. This was caused by an Android Studio version inclompatibility, resulting in Gradle not being able to setup the project. Resolving this issue, took more time than estimated. In retrospect, the has learned that setting up and learning to ise a new IDE is more time consuming, than first expected. 

\subsection{Server development}

The decision to implement the server in Dropwizard is reasonable as long as the server is used to provide a RESTful interface. There are other choices, but every solution usually includes some, or even all of the libraries used in Dropwizard. The server could be created from scratch and use a stand-alone HTTP server. This was not feasible in the time frame of this project, and would have been a waste of resources considering how well Dropwizard works.

\subsection{Device monitoring}
In the beginning the team designed a complete architecture with a server, Android app, and a monitoring device that communicated with a server. However, device monitoring turned out to be unrealistic within the project's time frame. The team decided to drop the feature entirely after consulting with the customer. As mentioned in section~\ref{sec:obtainingreq}, this requirement was optional.
After learning more about the CoSSMic project, the team realized that a solution for the complete architecture was already under development. This solution opened the possibility of interfacing to these devices, or at least design the server to be compatible with it. This information was unfortunately made available too late in the project. Because of this, the team was not able to include this functionality into the product. 
The team's device monitoring concepts is described in section ~\ref{sec:hardwareComponents}.


