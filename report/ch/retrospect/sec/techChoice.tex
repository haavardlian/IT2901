\section{Technical choices}
Even though the team aimed to plan well and make informed decisions before the development started, some changes had to be made on the fly. This section presents the reasoning behind both the planned decisions and the ones that was made during the development process.

\subsection{Architectural choices}
\subsubsection{Android development}
%A good deal of decisions made on the Android side of the project turned out to be reasonable. As only one of the team members had previous experience, pit falls could be avoided. The problem was that some of the Android building blocks are specific to the Android platform. Starting to use these building blocks can seem unintuitive unless you know about the alternatives and the problems these can cause. 
When it comes to the Android development, a great deal of the decisions made in that regard turned out to be reasonable. Few of the team members had previous experience with development in Android, which unfortunately led to a few pit falls that could have been avoided.

In retrospect, more time should have been devoted in the first couple of sprints not only to research, but also to learn more about how the platform works. A lot of time and resources could have been saved if the entire team had a clearer understanding of Android from the beginning of the project.

\subsubsection{Server side}

The decision to implement the server in Dropwizard is reasonable as long as the server is used to provide a RESTful interface. There are other choices, but every solution usually includes some, or even all of the libraries used in Dropwizard. One could also create the server from scratch and use a stand-alone HTTP server, but this was not feasible in the time frame of this project, and would have been a huge waste of resources considering how well Dropwizard works.

\subsubsection{In-house devices}
In the beginning the team designed a complete architecture with a server, Android app, and a monitoring device that communicated any new data to the server. This would have been a minimum viable complete solution. However, device monitoring turned out to be unrealistic within the project time frame. The team decided to drop the feature entirely after consulting with the customer. As mentioned in section~\ref{sec:obtainingreq}, this requirement was optional.

After learning more about the CoSSMic project, the team realized that a solution for the complete architecture already was under development. This solution opened the possibility of interfacing to these devices, or at least design the server to be compatible with it. This information was unfortunately not made available before halfway through the project, which in turn lead to not taking that solution into consideration during the development. Instead, further development involving this solution is documented in chapter~\ref{sec:further}.

\subsection{Choice of Android IDE}
The team agreed on using Android Studio after a brief discussion on whether to use Android Studio or Eclipse with ADT. Even though most of the team only had experience with Eclipse, it seemed like Android Studio had better features. This included better search function, auto completion of code, and integration to the Android tools.
 
At the start of the project, the team experienced some issues with Android Studio - some of the team members had trouble actually running the app. This was caused by an Android Studio update, resulting in Gradle not being able to setup the project. The issue took some time to solve, but after it was resolved, the team did not experience any more trouble with Android Studio. However, since Android Studio is a relatively new IDE, it still has some bugs and is updated frequently with major changes.

As the project progressed, it become clear that the learning curve was not as steep as first expected. Development would most likely not have been faster if Eclipse had been chosen.

%\subsection{Choice of GitHub for code management}
%By taking the customer's opinion in this matter into consideration, we were able to give the customer a better overview of the project and also to easily integrate our app into the customer's existing project. As a result, both the team and the customer were updated on the last version of the project and the team was more likely to receive relevant feedback. 
