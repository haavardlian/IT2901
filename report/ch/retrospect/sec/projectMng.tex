\section{Project management}
SECTION NOT FINISHED. As mentioned earlier, the team chose scrum. Yodiz, we now wish we had chose another scrum tool. After a while we added a person responsible for some serving on our work days. Also responsible for making sure we took brakes. 
The team introduced punishments for being to late for meetings and not doing all of your assignments. This was used on a team building with some food. Did this work optimally?
\subsection{Working with the customer}
SECTION NOT FINISHED. The team tried to have meetings with the customer once a week. This worked well and the team felt that it was a good thing that the customer participated to such an extent. Write something about the problems with the customer saying he is going to do something but did not and how we solved the problems around this?

\subsection{Team organization}
SECTION NOT FINISHED.
In the beginning of the project, the team distributed roles with different responsibilities. A project leader, a deputy project leader, a scrum master, a development responsible, a test responsible and a report responsible. Almost halfway through the project it became clear that the scrum master was also the one with the most Android experience. This resulted in a lot of extra work on the scrum master, while the deputy project leader had almost no work at all. Therefore, the team reflected over the project roles and the distribution and this resulted in a change of roles. The deputy project leader became the scrum master to distribute the resources in the best possible way. After this change the teams best Android resource could focus entirely on developing and helping others. The team was very satisfied with this decision and the resource distribution worked great after the change in roles. 

Was the responsibilities of the roles good?

\subsection{Choice of development methods}

The team chose to use scrum together with extreme programming. This process and why these were chosen is described in section ~\ref{sec:scrumDevProcess}. Even though Scrum is well known, many team members had different expectations to how the framework would be implemented in the project.


OMFORMULERES: Each team member spent a significant amount of time getting up to date with the Android platform and the front-end architecture of Android apps. In retrospect, more time should have been spent getting all group members up to date and practicing Android programming together in the beginning of the project. One of the team members had some experience with Android before and ended up spending a lot of time teaching and explaining to all of the other team members. If this had been organized in a way, for instance by holding a course in the beginning, some of the time spent on this could have been saved. 

\subsection{Choice of Scrum tool}