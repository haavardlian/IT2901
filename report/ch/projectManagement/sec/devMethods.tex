\section{Development methods}
\label{sec:devMethods}

As suggested by Sommerville~\cite{scrum}, agile development methods are usually a good fit for small developer teams when the goal is to produce a small or medium-sized product. These methods are also preferable when the system requirements are likely to be frequently changed.

The intention of such development methods is for the team to quickly deliver working software to the customer. This is beneficial because the customer may propose new features or change the requirements during the development process.

Based on the correlation between the descriptions of agile development methods and the team's needs, it seemed to be the most expedient to follow the guidelines of an agile development method. First and foremost it should be a development framework that was known to all team members, so that a minimum of time would be spent trying to learn a new process. As all of the team members had previous experience with Scrum, the obvious choice was to follow this approach. Scrum is a general agile method in which the main focus is the project management itself, rather than specific technical aspects.

In order to provide a more complete management framework for the project, it might be beneficial to use Scrum in a combination with a more technical agile approach.~\gls{Extreme programming} (XP) is such an approach.

In the following section, the focus will be on the concepts behind the Scrum process and the XP practices. The team's project process will be reviewed in detail in chapter~\ref{sec:devProcess}.