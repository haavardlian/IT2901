\subsection{Sprint 2}

\subsubsection{Sprint start}
In the second sprint the team focused on providing the customer with a working
prototype to further develop the requirements specification. Furthermore, a
working prototype might help the customer see how the team envisioned the app
and its functions. Lastly, work on the server application began to form the
foundation a framework of functions that the android app could use.

\subsection{Design}
Before implementing the specification that was approved by the customer, the
team built prototypes of the design in sketching programs to ensure a consistent
and clean design throughout the app. 

\subsection{Development}
As the GUI design neared an end, the team started implementing the features the
customer specifically asked to have included in the first iteration of the app.

\subsubsection{Server}
The groundwork for the apps server was put down in sprint 2. Using Dropwizard
the development team started the implementation of the functions available to
the app.

\subsection{Sprint end}

\subsubsection{Improper use of Scrum tool}
As mentioned in section~\ref{sec:scrumtool}, the team used a lot of time on
deciding on which Scrum tool to use for the project management. Although our
choice fell on Yodiz, the team was in lack of any previous experience with the
tool, and despite the team's efforts to get acquainted with the tool, a
misunderstanding arose and was not discovered until the end of the second
sprint.

The misunderstanding, displayed in figure~\ref{fig:wrongUse}, was that the team
assumed one could add multiple individuals as responsible on a particular task,
while Yodiz' functionality only assigned the time spent to the individual that
either created the task or was assigned as owner of the task.

As a result, it appeared as if only singular individuals performed the tasks,
even though the entire team in reality was participating, which was also
reflected in the burndown charts and the generated gantt diagram. 

To sort out this issue, the team went through old meeting reports and timesheets
to figure out which members of the team that had actually participated on the
particular task, and added new tasks and the time spent to the members that at
the time had not recorded this information, as shown in
figure~\ref{fig:addsTasks}.

This issue was unfortunate, but not insurmountable, and also not a critical
issue for the overall progress of the project.

\begin{figure}[H]
\includegraphics[width=\textwidth]{ch/sprints/fig/wrongUse.png}
\caption{Example screenshot to illustrate improper use of Yodiz}
\label{fig:wrongUse}
\end{figure}

\begin{figure}[H]
\includegraphics[width=\textwidth]{ch/sprints/fig/addsTasks.png}
\caption{Example screenshot to illustrate how the team resolved the issue.}
\label{fig:addsTasks}
\end{figure}

\subsubsection{Working away from the team}
One of the team members, Lars Erik, had to make a sudden trip abroad due to
death in the family. He stayed away from the team for a total for 12 days. Prior
to departure the team divided the tasks with a lighter workload assigned to Lars
Erik, as described in the risk analysis. Even with the redistributed workload
the team and Lars Erik in particular found the distance and time difference to
be a bigger problem than expected. Some work was left undone and had to be
picked up at the end of the sprint or pushed to the next sprint. The team
learned that the impact of such leaves of absence should probably be
overestimated rather than underestimated in the eventuality of another event.

\subsubsection{General results from sprint 2} The customer was happy with
progress on the prototype, and provided rich and detailed feedback on how the
app should be improved for the next iteration. The sprint was deemed successful
by the team members.

\subsubsection{Sprint backlog}

The backlog and time usage result.

\begin{table}[H]
    %\resizebox{\textwidth}{!}{
    \begin{tabular}{|l|p{4cm}|c|c|r}%
    \hline \bfseries User story & \bfseries Details & \bfseries Hours estimated & \bfseries Hours spent & \bfseries Hours left
    \csvreader[head to column names]{ch/projectManagement/sec/sprint2/userstories.csv}{}% use head of csv as column names
    {\\\hline \id & \title & \estimated & \spent & \left}% specify your coloumns here
    %\hline
    \end{tabular}
    \caption{Sprint 2 backlog}
%}
\end{table}
