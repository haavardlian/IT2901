\subsection{Extreme Programming}
The XP approach has several practices and values we have practiced throughout the duration of the project. These practices, and the modifications we made to them in order to adjust this approach to our project, are described in the following sections.

\subsubsection{Pair programming}
The pair programming practice implies that developers work in pairs and continiously checks each other's work. We have practiced this principle by dividing the team into small groups or pairs when designing prototypes, deciding on architecture and in a few cases, collaborated on code sections.


\subsubsection{Planning game}
The team have interpreted the planning game in XP to be a more generic description of planning methods. We therefore argue that planning poker, which the team have used for estimating the tasks, is a more specific implementation of the planning game in XP. Amongst other, one of the most important aspects of planning poker is that no team member should not know what the rest of the team choose to estimate in order to avoid the influence of the other participants. This is not mentioned in the planning game specification in XP.

\subsubsection{Continious integration and collective ownership}
The continious integration practice implies that as soon as a task is completed, it is to be integrated into the whole system. The team has practiced this principle by working on code on separate branches in Git and merged these branches in a master-branch when a task was considered to be completed.

The team has also practiced the collective ownership principle by sharing and collaborating on the project's code on GitHub. In addition, we agreed with the customer on following the Java code convention and Android code convention in the development.

\subsubsection{Design improvement and optimization}
The object of this practice is to continiously refactor code as soon as possible code improvements are found and to leave the optimization\gls{opt} process to last. 

The team has continiously followed this practice and also practiced it during design improvements in the development of prototypes.
 
\subsubsection{Small releases}
The object of this practice is to develop a minimal useful set of functionality first. Releases of the system are frequent and incrementally add functionality to the first release. The team has followed this practice by having sprints with a duration of two weeks, with a release for each sprint.


\subsubsection{Simple design}
In this practice, it is emphasized that the only design that is to be carried out, is the design that meets the current requirements.

Although the team worked iteratively and in close collaboration with the customer, it has not always been easy to \emph{only} implement the basic functionality the customer requested. The initial assignment was very open for interpretation, and because it was the team's task to define the requirements specification, it was hard not to try to add functionality that we sincerly thought would improve the application, even though that functionality might not had been marked as a high priority assignment.


\subsubsection{Sustainable pace}
This practicee implies avoiding large amounts of overtime, as it is likely to result in reduction of the code quality and less productivity. The team has practiced this by having an estimation process where the amount of work hours were set to twenty hours a week.


\subsubsection{On-site customer}
In this practice, the customer should be available full time for the use of the team. Usually, the customer is a member of the development team and is responsible for bringing system requirements to the team for implementation.

This practice does not apply to the team's project, as our customer comes from an external organization. However, we were co-located with our customer at a regular basis, usually once a week and hence got continious feedback and acceptance testing.

\subsubsection{Test-first development}
In XP, an automated unit test framework is used to write tests for a new piece of functionality before that functionality itself is implemented. The team has not followed this practice.

Although we initially thought it would be a good idea to have a test-driven development approach towards the testing process, the team quickly realised that we both lacked the experience and the time to follow this practice and therefore chose to drop it.