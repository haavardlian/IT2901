\subsection{Extreme Programming}
The XP approach has several principles the team practiced throughout the project. A brief summary of these principles are given in table~\ref{tab:exProg}. Further details on how these principles were used may be found in section~\ref{sec:adapExtremeProgr}.

\begin{table}[H]
\rowcolors{0}{darkgray}{lightgray}
\begin{tabular}{|p{4cm}|p{11.7cm}|}
\hline
\textbf{Principle} & \textbf{Description} \\\hline
Pair programming & Developers work in pairs and continuously checks each other's work.\\\hline
Planning game & Involves the whole team in the planning process. The plan is developed incrementally and, if problems should arise, adjusted so the software functionality is reduced instead of delaying the delivery.
\\\hline
Continuous integration and collective ownership& As soon as a task is completed, it is to be integrated into the whole system.\\\hline
Design improvement and optimization & Continuously refactor code as soon as code improvements are found and leave the optimization process to last. \\\hline
Small releases & Develop a minimal useful set of functionality first. New releases of the system are frequent, and functionality is incrementally added to the first release.\\\hline
Simple design & The only design that is to be carried out is the design that meets the current requirements.\\\hline
Sustainable pace & Avoid large amounts of overtime as it is likely to result in reduction of the code quality and less productivity. \\\hline
Test-first development & An automated unit test framework is used to write tests for a new piece of functionality before that functionality itself is implemented. \\\hline
On-site customer & The customer should be available full time for the team. The customer is usually a member of the development team and is responsible for bringing in system requirements for the implementation.\\\hline
\end{tabular}
\caption{Principles in extreme programming}
\label{tab:exProg}
\end{table}
