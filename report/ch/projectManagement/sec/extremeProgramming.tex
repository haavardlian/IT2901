\subsection{Extreme Programming}
The XP approach has several principles the team practiced throughout the duration of the project. A brief summary of these principles are given in table~\ref{tab:exProg}. In the development process chapter it is described how the team used these principles in the project and the modifications that were made in order to adjust this approach to the project.

\begin{table}[H]
\rowcolors{0}{darkgray}{lightgray}
\begin{tabular}{|p{4cm}|p{11.7cm}|}
\hline
\textbf{Principle} & \textbf{Description} \\\hline
Pair programming & Developers work in pairs and continuously checks each other's work.\\\hline
Planning game & Involves the whole team in the planning process. The plan is developed incrementally and, if problems should arise, adjusted so the software functionality is reduced instead of delaying the delivery.
\\\hline
Continuous integration and collective ownership& As soon as a task is completed, it is to be integrated into the whole system\\\hline
Design improvement and optimization & Continuously refactor code as soon as possible code improvements are found and to leave the \gls{opt} process to last. \\\hline
Small releases & Develop a minimal useful set of functionality first. Releases of the system are frequent and incrementally add functionality to the first release.\\\hline
Simple design & The only design that is to be carried out is the design that meets the current requirements\\\hline
Sustainable pace & Avoiding large amounts of overtime, as it is likely to result in reduction of the code quality and less productivity. \\\hline
Test-first development & An automated unit test framework is used to write tests for a new piece of functionality before that functionality itself is implemented. \\\hline
On-site customer & The customer should be available full time for the use of the team. The customer is usually a member of the development team and is responsible for bringing system requirements to the team for implementation.\\\hline
\end{tabular}
\caption{Principles in extreme programming}
\label{tab:exProg}
\end{table}