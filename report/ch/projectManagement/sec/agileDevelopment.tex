\section{Agile software development}

Agile development methods~\cite{scrum} are known to be successful for small developer teams where the goal is to produce a small or medium-sized product, and is best suited for application development where the system requirements are likely to be frequently changed.

The intention of such development methods is that the team is to deliver working software quickly to the customer so that he may propose new features or change the requirements for future iterations of the application.

Based on the correlation between the descriptions of agile development methods and the team's needs, it seemed to be the most expedient and beneficial for the team to follow the guidelines of an agile development framework. First and foremost it should also be a development framework that is known to the team, so that a minimum of time will be spent trying to learn a new process.

As all of the team members had previous experience with the Scrum approach from the course IT1901, a mandatory course in the computer science bachelor at NTNU, the obvious choice was to follow this approach, which is a general agile method in which the main focus is the project management itself, rather than specific technical aspects.

In order to provide a management framework for the project, it may be beneficial to use Scrum in a combination with a more technical agile approach. Extreme programming (XP) is such an approach.

In the following section, we will focus on why we chose to use the Scrum approach in combination with XP, the concept behind the Scrum process and the XP practices we adopted into the project. \todo[inline]{\textbf{Endre denne referansen? }The team's project process will be reviewed in detail in chapter~\ref{sec:sprints}.}