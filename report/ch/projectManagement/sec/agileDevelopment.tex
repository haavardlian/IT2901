\section{Development methods}

As suggested by Sommerville~\cite{scrum}, agile development methods are usually a good fit for small developer teams when the goal is to produce a small or medium-sized product, and the system requirements are likely to be frequently changed.

The intention of such development methods is for the team to deliver working software quickly to the customer so that he may propose new features or change the requirements for future iterations of the product.

Based on the correlation between the descriptions of agile development methods and the team's needs, it seemed to be the most expedient for the team to follow the guidelines of an agile development method. First and foremost it should also be a development framework that is known to the team, so that a minimum of time will be spent trying to learn a new process. As all of the team members had previous experience with Scrum, the obvious choice was to follow this approach, which is a general agile method in which the main focus is the project management itself, rather than specific technical aspects.

In order to provide a management framework for the project, it may be beneficial to use Scrum in a combination with a more technical agile approach. Extreme programming (XP) is such an approach.

In the following section, we will focus on why we chose to use the Scrum approach in combination with XP, the concept behind the Scrum process and the XP practices that was adopted into the project. The team's project process will be reviewed in detail in chapter~\ref{sec:sprintOverview}.