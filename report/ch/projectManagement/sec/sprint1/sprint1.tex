\subsection{Sprint 1}

\subsubsection{Sprint start}

Sprint 1 was the first actual sprint that the team had. The goals that were set for the sprint was to get a meeting with the customer,
get a better grip of what the tasks ahead was, continuous research about the task and set all the team members roles.

\subsubsection{Sprint burndown}

The sprint burndown chart~\ref{fig:sprint1burndown} clearly shows that no progress was made. But the team had yet to start use yodiz properly.

\begin{figure}[H]
\includegraphics[width=\textwidth]{ch/projectManagement/fig/sprint1burndown.png}
\caption{Sprint 1 burndown chart}
\label{fig:sprint1burndown}
\end{figure}

\subsubsection{Sprint backlog}

The backlog and time usage result.

\begin{table}[H]
\begin{tabular}{|l|p{4cm}|c|r}%
    \hline \bfseries User story & \bfseries Details & \bfseries Hours estimated & \bfseries Hours spent
    \csvreader[head to column names]{ch/projectManagement/sec/sprint1/sprint1userstories.csv}{}% use head of csv as column names
    {\\\hline \id & \title & \estimated & \spent}% specify your coloumns here
\end{tabular}
\end{table}

\subsubsection{Sprint end}
The main focus of this sprint was the project management part. The scrum model was chosen as project methodology. 
A rough project plan was made, and concepts for the application was made.
The customer was very pleased with the teams ideas and concepts for the app, and approved of the temporary specification. 
The team began writing documentation in \LaTeX. To uphold the scrum model, the tool Yodiz was chosen. More about Yodiz can be found in section~\ref{sec:yodiz}.

The customer wanted the application to be developed at the Android platform. Android-studio became the tool of choice for this.

The team members deemed sprint 1 very successful.

