\section{Acceptance testing}
Acceptance testing is a form of testing that checks whether or not the functional requirements are met. Acceptance testing is normally done in cooperation with the customer. During the development of the application acceptance testing was done regularly, often once or twice each sprint. This gave the team valuable feedback on the applications status and what parts could be improved and where to focus the time and resources until the next meeting.

At the end of the development process the customer met with the team to review the finished product. The customer previewed the application while the team answered questions from the customer. In the end the customer was satisfied with the product and asked that the team make the documentation available on the GitHub repository to simplify further development.