\section{Usability testing}
\label{sec:userTest}
In order to make our app as user friendly as possible the team decided to conduct several usability tests on the app in different stages.

Usability testing is a technique used to evaluate a product by testing it on actual users. Because usability testing gives a direct input on how real users actually use a system, this usability practice can be considered irreplaceable.~\cite{usability}. In our conduction of the usability tests, we used two different methods, as described in the subsequent sections.



\subsection{Hallway testing}

\subsection{Remote usability testing}
When the developers and test subjects find themselves separated by long distance, conducting a usability test might arise some unforeseen challenges.

For instance, the remote testing may lack the immediacy and sense of ''presence'' that often is desired to support a collaborative testing process. It may also include having reduced control over the testing environment~\cite{remoteTest}. 

The team chose this testing method because we wanted to get input from users from different age groups and that might not have the same technical expertise as most of the users we suspected to run into at the university. As it was mentioned by the customer that the app should be so easy to use that anyone should be able use it, we considered remote usability to be the best testing method to achieve this.

As the team members knew few users that fulfilled the requirements of lack of technical expertise and different age spans, we contacted friends and relatives and asked if they could be our test subjects. The results from these test can be found in section x.

