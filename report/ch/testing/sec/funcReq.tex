\newpage
\section{Testing functional requirements}
\label{sec:funcTest}
Functional tests are tests that inspect the functionality of the software. The internal classes and methods are tested indirectly. Functional testing is a type of black-box testing~\cite{blackbox}, which is a software testing method that examines application functionality without exploring the application's internal structures. This is done by feeding the application data and checking that it responds properly. The given input is either a) correct data, where it is checked that the data is handled correctly, or b) incorrect data, to make sure the application fails gracefully and does not crash. Should a test fail, there is a bug in the code that needs to be corrected.

To test a system the following steps are required:\\\\
1. Identify what features or functionality to test\\
2. Specify input to the system\\
3. Specify expected output from the system\\
4. Execute

\subsection{Test table}
Table \ref{tab:testTable} presents a list of functionality that should be tested thoroughly in every iteration of the application if applicable. Because of the lack of automatic unit testing, the functionality had to be tested manually in order to be sure that new functionality and bug fixes did not break any existing functionality.

\begin{longtable}{|l|p{12.5cm}|l|}

\hline
\multicolumn{1}{|l|}{\textbf{ID}} &  \multicolumn{1}{c|}{\textbf{Description}} &\multicolumn{1}{l|}{Result}\\
\endfirsthead


\multicolumn{3}{c}%
{{\bfseries \tablename\ \thetable{} -- continued from previous page}} \\\hline
\multicolumn{1}{|l|}{\textbf{ID}} &  \multicolumn{1}{c|}{\textbf{Description}} & \multicolumn{1}{l|}{Result}\\
\endhead

\multicolumn{3}{|l|}{{Continued on next page}} \\ \hline
\endfoot
 
\endlastfoot

\hline
FR1.1. &The user should be able to see how much electricity each device in his house uses on an average basis, on a monthly basis, and on a yearly basis.&X\\\hline
FR1.2. &The user should be able to see how much electricity each device in his house produces on an average basis, on a monthly basis, and on a yearly basis.&X\\\hline
FR1.3. &The user should be able to add new devices, and to specify whether a device produces or consumes power for each device.&X\\\hline
FR1.4. &The user should be able to add the amount of power consumed.&X\\\hline
FR1.5. &The user should be able to add the amount of power produced.&X\\\hline
FR2.1. &The app should show the user graphs that show his usage over a predefined set time frames chosen by the user.&X\\\hline
FR2.2. &The app should display a graph detailing the selected devices energy consumption.&X\\\hline
FR2.3. &The app should display a graph comparing the energy consumption from selected devices.&X\\\hline
FR3.1. &The user should have a list over the most used and best rated energy saving tips.&X\\\hline
FR3.2. &The user should be able to rate energy saving tips in a rating bar.&X\\\hline
FR3.3. &The user should have a list over what energy saving tips he has done.&X\\\hline
FR3.4. &The user will also be able to add energy saving tips he want to do in the future.&X\\\hline
FR3.5. &The user should be able to add new energy saving tips to the global list of energy saving tips.&X\\\hline
FR4.2. &The user should be able to connect his Facebook profile to the app. Fetch data from Facebook.&X\\\hline
FR4.4. &The user should be able to choose which criteria he wants to use when comparing his energy usage with other people.&X\\\hline
FR5.1. &The user should be able to share graphs on Facebook.&X\\\hline
FR5.2. &The user should be able to share energy saving tips on Facebook.&X\\\hline
FR5.3. &The app should contain a list with Facebook friends that uses the app.&X\\\hline
FR6.1. &The app should be written so it is easy to add new languages.&X\\\hline
FR6.2. &The app should use the language used on the Android device. The default language, if the Android language is not set, is English.&X\\\hline
\caption{Table showing testing of functional requirements}
\label{tab:testTable}
\end{longtable} 

\begin{comment}
\begin{table}[H]
\centering
\rowcolors{0}{darkgray}{lightgray}
\begin{tabular}{|l|l|}
\textbf{Description} & \textbf{Result}\\\hline 
\arrayrulecolor{lightgray}
Add device&x\\ \cline{1-2}
Remove device&x\\ \cline{1-2}
Edit device&x\\ \cline{1-2}
Add power used for device&x\\ \cline{1-2}
View power usage&x\\ \cline{1-2}
View power distribution&x\\ \cline{1-2}
Create new tip&x\\ \cline{1-2}
Log into Facebook&x\\ \cline{1-2}
Log out of Facebook&x\\\arrayrulecolor{black}
\hline
\end{tabular}
\caption{Test table for functional tests of app}
\label{tab:testTable}
\end{table}
\end{comment}
