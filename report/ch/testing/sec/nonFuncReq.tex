\section{Testing non-functional requirements}
\label{sec:testingnonfunctionalrequirements}
Non-functional requirements are requirements that represent the overall usability and quality of the app. To test these requirements, different kinds of usability tests were performed.

\subsection{Usability testing}
\label{sec:userTest}
Usability testing is a technique used to evaluate an app by testing it on actual users, preferably within the applications target audience. This kind of test give the developers a better understanding of how a common user reacts to specific features in the app. If the test subjects struggle with, or are unable to complete certain tasks, some changes in the user interface might be necessary.
The target user group for the app is home owners. This is a wide target audience that sets extra requirements on the app usability. To test this usability two different methods were used, as described in the subsequent sections.

\subsubsection{Hallway testing}
Hallway testing is a typical method for usability testing~\cite{hallwaytesting}. In hallway testing, the subjects are random people (''people who pass by in the hallway,'' hence the name). During the test, the subjects are given a set of tasks to perform while the developers observe how they perform their given tasks. This allows the developers to make some choices early in the development process without the need for protracted user testing with carefully selected users. The team used this method to some degree, mostly in the early stages of prototyping.

\subsubsection{Remote usability testing}
By using remote usability testing~\cite{remoteTest}, the team was able to get input and opinions from users closer to the app's target audience.
The tests were conducted mainly on friends and relatives. The subjects were given the app and a set of tasks. After the tests, the subject filled out a simple form where they rated the usability of the app. The form was then used to make improvements on the app user-interface.

\subsection{Performing the tests}
The project had two main requirements to cover the overall usability of the app. The following sections describes how these were tested.

\subsubsection{NFR1: The app should be easy to use}
The target audience of this app does not necessarily have any technical insight. This means that the app should to be easy to use for the average person. This was tested by performing both hallway testing and remote usability testing.

\subsubsection{NFR2: The app should follow standard Android specifications}
The target platform for the app is Android, which has a set of standards aiming to make Android apps consistent. By running the usability test on Android users, the team got feedback from the test subjects on whether the app looked and behaved like an Android app.

\subsubsection{Test plan}
When a usability test is performed, the subjects are usually given a set of tasks to perform. During the test, the developers should take special note of whether the subjects are struggling or gets frustrated. The usability test tasks are presented in table~\ref{tab:usabilityTests}. For each task, the testers were asked the following questions:\\\\
1. What did you find complicating about the tasks given above?\\
2. Which of the above tasks did you find intuitive and easy to perform?\\
3. What features do you think should be added to make the tasks above easier?\\\\
The test resultes were used to improve the user interface. An example feedback document can be found in appendix~\ref{sec:userTestExample}.

\begin{table}[H]
\rowcolors{0}{darkgray}{lightgray}
\centering
\begin{tabular}{|l|p{10.5cm}|}
\hline
\textbf{Task id }& \textbf{Task description}\\\hline
\textbf{1.}&\textbf{Basics}\\\hline
1.1 & Log in\\\hline
1.2 & View your profile\\\hline
1.3 & Add a new residence\\\hline
1.4 & Log out\\\hline
\textbf{2.}&\textbf{Devices}\\\hline
2.1 & Add a new device\\\hline
2.2 & Add usage to the device\\\hline
\textbf{3.}&\textbf{Usage}\\\hline
3.1 & Check the usage of the device called "Heater"\\\hline
3.2 & Compare the usage of your radio and TV\\\hline
\textbf{4.}&\textbf{Comparison}\\\hline
4.1 & Compare your usage with a friends\\\hline
4.2 & Compare your usage with someone with a similar profile\\\hline
\textbf{5.}&\textbf{Tips}\\\hline
5.1 & Add a tip to "my tips"\\\hline
5.2 & View your list of tips\\\hline
5.3 & Mark a tip as done\\\hline
\end{tabular}
\caption{Overview of usability test tasks}
\label{tab:usabilityTests}
\end{table}