\section{Testing non-functional requirements}
\label{sec:testingnonfunctionalrequirements}
Non-functional requirements are requirements representing the overall usability and quality of the application. One requirement is that the application should be easy to use and user friendly. To test this different kinds of usability tests where used.
\todo{kanskje skrive noe mer om hvordan vi vil teste de andre ikke-funksjonelle kravene?}

\subsection{Usability testing}
\label{sec:userTest}

Usability testing is a technique used to evaluate an application by testing it on actual users, preferable within the applications target audience and with different backgrounds and a different level of technical understanding. This kind of tests gives the developers a better understanding of how the common users react to specific tasks in the application, if they struggle with certain tasks, and if they expect something to behave differently.
The user base for the application is more or less everyone, but the functionality mainly targets home owners. This is a wide target audience that puts serious requirements on the applications usability. To test this usability we used two different methods, as described in the subsequent sections.

\subsubsection{Test methods in usability testing}

\paragraph{Hallway testing}
Hallway testing is a common method for usability testing early in the applications development. The meaning behind the name "Hallway testing" is that the developer walks out to the hallway, asks a random person to test the application, and observes how the person handles certain tasks. This allows the developers to make certain choices early in the development process without the need for proper user testing with selected users. 
The team used this method to some degree, mostly in the early stages of prototyping.

\paragraph{Remote usability testing}
When the developers and test subjects find themselves separated by long distance, conducting a usability test might arise some unforeseen challenges.

For instance, the remote testing may lack the immediacy and sense of ''presence'' that often is desired to support a collaborative testing process. It may also include having reduced control over the testing environment~\cite{remoteTest}. 

Using remote usability testing the team was able to get input and opinions from users closer to the applications target audience.
The tests were conducted mainly on friends and relatives. The testers were given the application and a set of tasks. When they were done, they filled out a simple form where they rated the usability of the application. The form was then used to make improvements on the applications user-interface.

\subsubsection{Test plan}
When performing a usability test it is common to present the subject with a set of task that they will perform. During this tasks the developers should take special note when the subject is struggling or is starting to get frustrated. When testing the application this tasks were given:
\todo{legg til de faktiske spørsmålene vi brukte under brukertesting}