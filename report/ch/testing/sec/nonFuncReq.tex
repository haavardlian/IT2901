\section{Testing non-functional requirements}
\label{sec:testingnonfunctionalrequirements}
Non-functional requirements are requirements representing the overall usability and quality of the application. To test these requirements different kinds of usability tests was performed.

\subsection{Usability testing}
\label{sec:userTest}

Usability testing is a technique used to evaluate an application by testing it on actual users, preferable within the applications target audience and with different backgrounds and a different level of technical understanding. This kind of tests gives the developers a better understanding of how the common users react to specific tasks in the application. If they struggle with, or are unable to complete certain tasks, some interface changes might be necessary.
The user base for the application is more or less everyone, but the functionality mainly targets home owners. This is a wide target audience that puts serious requirements on the applications usability. To test this usability we used two different methods, as described in the subsequent sections.

\subsubsection{Hallway testing}
Hallway testing is a typical method for usability testing~\cite{hallwaytesting}. In hallway testing, the testers are a small group of random people (''people who pass by in the hallway,'' hence the name). During the test, the testers are given a set of tasks to perform, while the developers observes how they handle their given tasks. This allows the developers to make some choices early in the development process without the need for protracted user testing with carefully selected users. 
The team used this method to some degree, mostly in the early stages of prototyping.


\subsubsection{Remote usability testing}
%When the developers and test subjects find themselves separated by long distance, conducting a usability test might arise some unforeseen challenges.

%For instance, the remote testing may lack the immediacy and sense of ''presence'' that often is desired to support a collaborative testing process. It may also include having reduced control over the testing environment

By using remote usability testing~\cite{remoteTest}, the team was able to get input and opinions from users closer to the applications target audience.
The tests were conducted mainly on friends and relatives. The testers were given the application and a set of tasks. When the tests had been performed, the testers filled out a simple form where they rated the usability of the application. The form was then used to make improvements on the applications user-interface.

\subsection{What was tested}
The project ended up with three main requirements to cover the overall usability of the application. The following paragraphs describes how they were tested.

\paragraph{NFR1: The application should be easy to use}
The target audience of this application does not necessary have any technical insight, this means the application needs to be easy to use for the average person. This was tested by using normal usability testing and remote usability testing.

\paragraph{NFR2: The application's user interface should follow standard Android specifications for graphical user interfaces}
The target platform for the application is android, and it has a set of standards and rules for making applications consistent.
By running usability test on android users, the team got information from the subjects on, for instance, how natural the application felt.

\paragraph{NFR3: The application should be able to function without the purchase of new hardware}
\todo{Fjerne?}


\subsubsection{Test plan}
When performing a usability test it is common to present the subject with a set of task that they will perform. During this tasks the developers should take special note when the subject is struggling or is starting to get frustrated. The tasks given is presented in table \ref{tab:usabilityTests}
\todo{Referanse til test resultater?}

\begin{table}[H]
\rowcolors{0}{darkgray}{lightgray}
\begin{tabular}{|l|p{10.5cm}|}
\hline
\textbf{Task id }& \textbf{Task description}\\\hline
\textbf{Basics }&\\\hline
1.1 & Log in\\\hline
1.2 & View your profile\\\hline
1.3 & Add a new residence\\\hline
1.4 & Log out\\\hline
\textbf{Devices}&\\\hline
2.1 & Add a new device\\\hline
2.2 & Add usage to the device\\\hline
\textbf{Usage}&\\\hline
3.1 & Check the usage of the device called "Heater"\\\hline
3.2 & Compare the usage of your radio and TV\\\hline
\textbf{Comparison}&\\\hline
4.1 & Compare your usage with a friends\\\hline
4.2 & Compare your usage with someone with a similar profile\\\hline
\textbf{Tips}&\\\hline
5.1 & Add a tip to "my tips"\\\hline
5.2 & View your list of tips\\\hline
5.3 & Mark a tip as done\\\hline
\end{tabular}
\caption{Usability test tasks}
\label{tab:usabilityTests}
\end{table}