\section{Testing non-functional requirements}
\label{sec:testingnonfunctionalrequirements}
Non-functional requirements are not connected to specific functionality but more the overall usability and quality of the application. To test non-functional requirements different kinds of usability tests where used.
\todo{kanskje skrive noe mer om hvordan vi vil teste de andre ikke-funksjonelle kravene?}

\subsection{Usability testing}
\label{sec:userTest}

Usability testing is a technique used to evaluate an application by testing it on actual users, preferable within the applications target audience and with different backgrounds and technical knowledge. This kind of tests gives the developers a better understanding of how the common users react to specific challenges in the application, and how they expect things to work.
In our conduction of the usability tests, we used two different methods, as described in the subsequent sections.

\subsubsection{Test methods in usability testing}

\paragraph{Hallway testing}
Hallway testing is a common method for usability testing early in the applications development. The meaning behind the name "Hallway testing" is that the developer walks out to the hallway, asks a random person to test the application. 

\paragraph{Remote usability testing}
When the developers and test subjects find themselves separated by long distance, conducting a usability test might arise some unforeseen challenges.

For instance, the remote testing may lack the immediacy and sense of ''presence'' that often is desired to support a collaborative testing process. It may also include having reduced control over the testing environment~\cite{remoteTest}. 

The team chose this testing method because we wanted to get input from users from different age groups and that might not have the same technical expertise as most of the users we suspected to run into at the university. As it was mentioned by the customer that the app should be so easy to use that anyone should be able use it, we considered remote usability to be the best testing method to achieve this.

As the team members knew few users that fulfilled the requirements of lack of technical expertise and different age spans, we contacted friends and relatives and asked if they could be our test subjects. The results from these test can be found in section x.

\subsubsection{Test plan}
\todo{legg til de faktiske spørsmålene vi brukte under brukertesting}