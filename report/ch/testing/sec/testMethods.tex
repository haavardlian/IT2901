\section{Test methods}
Testing was done with several approaches, which helped structuring the work involved with testing. The team decided to use functional testing and acceptance testing in order to test the server and the Android application during development. In addition, the customer wanted to be closely integrated with the development of the Android application the assure that possible mistakes and misconceptions were discovered as early as possible. The team chose to call this iterative customer testing.

\subsection{Iterative customer testing}
From the very beginning of the design phase, the customer met with the team at least once per sprint, and often twice. From approving and giving feedback on paper prototypes to comments on the final design, the customer was tightly integrated to the development of the application. Each or every other Friday, the customer would give his thoughts on the the current development prototype of the application. Comments on earlier request and further thoughts were noted and taken into consideration when the team had sprint start up meetings. 

\subsection{Functional testing}
\label{sec:funcTest}
Functional testing are tests that control the functionality of the software. Internal classes and methods is tested indirectly. This is a type of Black Box Testing~\cite{blackbox}. \todo[inline]{The team chose several classes and methods that the application should complete, and some that it should not, but handle in a satisfying way.} This was done by asking the application to do something and compare the result with what was expected.

The following steps are required in order to test the system
\begin{enumerate}
\item Identify what features are to be tested
\item Specify input to the system
\item Specify expected output from the system
\item Execute. Feed input data into the system and compare the output with what is expected. If the output is not what was expected there is a bug that needs to be investigated and fixed before the test is run again.
\end{enumerate}

\subsection{Test table}
\begin{table}[H]
\begin{tabular}{|l|l|l|}
\hline
\rowcolor{darkgray} & \textbf{Description} & \textbf{Result}\\\hline \arrayrulecolor{lightgray}
\rowcolor{lightgray} Server& \textit{Functional tests of server}&\\
\cline{1-2}\cline{2-3}
\rowcolor{lightgray}&Server should deliver data in JSON format&\\
\cline{1-2}\cline{2-3}
\rowcolor{lightgray}&\textit{REST calls that should have a reply}&\\
\cline{1-2}\cline{2-3}
\rowcolor{lightgray}&Fetch usage data for user&\\
\cline{1-2}\cline{2-3}
\rowcolor{lightgray}&Fetch usage data for users friends?&\\
\cline{1-2}\cline{2-3}\arrayrulecolor{black}
\rowcolor{lightgray}&Store usage data for user&\\\hline
\arrayrulecolor{darkgray}
\rowcolor{darkgray}Android application & \textit{Functional tests of application} &  \\ \cline{1-2}\cline{2-3}
\rowcolor{darkgray}&Add device&\\ \cline{1-2}\cline{2-3}
\rowcolor{darkgray}&Remove device&\\ \cline{1-2}\cline{2-3}
\rowcolor{darkgray}&Add power used for device&\\ \cline{1-2}\cline{2-3}
\rowcolor{darkgray}&Add power produced for device&\\\arrayrulecolor{black}
\hline
\end{tabular}
\caption{Test table}
\label{tab:testTable}
\end{table}

\subsection{Planned tests}
\begin{table}[H]
\centering
\rowcolors{1}{darkgray}{lightgray}
\begin{tabular}{|l|l|}
\hline
Test 1&Get usage data in JSON for user 1\\
Test 2&Store usage data in database for user 1\\
Test 3&User can read the amount of power a device has produced\\
Test 4&User can add a device\\
Test 5&User can delete a device\\
Test 6&User can read the amount of power a device has consumed\\
Test 7&User can type in the amount of energy produced by a device\\
Test 8&user can type in the amount of energy consumed by a device\\
\hline
\end{tabular}
\caption{Planned tests}
\label{tab:plannedTable}
\end{table}

\subsection{Acceptance testing}
In addition to the functional testing, the customer also participated in the testing process. This was done because the customer had insight in his requests and could provide \todo{a fresh set of eyes to spot problems}. The customer also delivered the application to third party users so the application could be tested in real life circumstances.

Acceptance testing was done throughout the development process, and was also a part of the final testing.

 
\subsection{Testing non-functional requirements}
\label{sec:testingnonfunctionalrequirements}
Testing the non-functional requirements is a process quite different from testing that functionality has been implemented correctly.

The only non-functional requirements the team will be testing is the usability of the Android application. These test will primarily be 
tested by individuals that are not a part of the development team. The team will have a user test, or a multiple of such tests, to observe how people use and experience the application without interference from the team. 

\todo[inline]{Fikse dette avsnittet}
After testing the users will be able to rate different aspects of the application and comment on usability. Detailed 
test plans for testing usability can be found in chapter 3.x Research: Testing.