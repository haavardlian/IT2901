\chapter{Testing}

In order to deliver a complete and functional product to the customer, it is paramount to make sure the system is regularly tested during development. 
By running tests we increase system stability and robustness, which in turn increases the usability of the product. 

There are several approaches when it comes to testing of software systems. The team looked into unit testing and functional testing. \todo{\textbf{Put in glossary:}
Unit testing is a form of white box testing.} When unit testing Java based applications it is very common to use JUnit~\cite{junit}.
With JUnit you can easily test every function on the program. The team decided that functional testing made the most sense to use for this project. See section \todo{add ref} for further descriptions.

At the end of each sprint tests would be runned to see whether the applications met the specifications. Tests would also be runned at the end of the development process. 

In addition to this, the customer also participated in the testing, and delivered the application to third-party users so the application will be user-tested as well.
\newpage
\input ch/testing/sec/testMethods.tex