\section{Project methodology}

\label{sec:scrumDevProcess}
As can be seen in section ~\ref{sec:scrumProjectManagement}, the team chose Scrum as the project management framework. This section aims to explain how and why Scrum was used and what issues that were encountered. It also describes how XP was adapted by the team members and issues in relation to this.

\subsection{Our adaptation of Scrum}
Since scrum is a framework it is easily modified to fit the needs of the project it is applied to. The team made several modifications and adaptions to the scrum framework. 

\subsubsection{Overall modifications}
The concept behind the Scrum approach is to empower the entire team to make decisions, so that a project leader would become redundant. However, in order to manage and make the best of the team's resources, the team found it necessary to keep this position in this project. The tasks related to this role is described in table ~\ref{tab:mainResponsibilities}

\subsubsection{Sprints}
The team chose to have sprints with a duration of two weeks. Unfortunately, having daily meetings was not possible with the team's schedule. The team came to an agreement to meet twice a week, each beginning with a daily meeting. In addition, the team has a fixed time to meet with the customer once a week, unless the team or the customer does not see the need for it.

\subsubsection{Scrum Tool}
The team's main requirements in a Scrum tool was that it should be easy to use, effective for project management, had Git integration, and that it could create charts for documentation purposes. With this tool the team wanted to be able to get an overview over the projects progress and the team's performance. The team chose Yodiz \cite{yodiz} because it provides free accounts for educational purposes and satisfied the team's requirements.


\subsubsection{Planningpoker}
In the team's planning poker sessions, it was decided to use the units small (S), medium (M), large (L) and extra large (XL), each of them respectively representing one hour (S), two hours (M), four hours (L) and eight hours(XL).


\subsection{Our adaptation of Extreme Programming}
Several modifications and adaptations were made to the Extreme Programming method in order to optimize it to fit the team's needs.

\paragraph{Pair programming}
The team has practiced this principle by dividing the team into small groups or pairs when designing prototypes, deciding on architecture and in a few cases, collaborated on code sections.

\paragraph{Planning game}
The team interpreted the planning game in XP to be a more generic description of planning methods. \todo{We therefore argue that planning poker, which the team used for estimating the tasks, is a more specific implementation of the planning game in XP.} Amongst other, one of the most important aspects of planning poker is that no team member should not know what the rest of the team choose to estimate in order to avoid the influence of the other participants. This aspect is not mentioned in the planning game specification in XP.

\paragraph{Continuous integration and collective ownership}
The team has practiced this principle by working on code on separate branches in Git and merged these branches in a master-branch when a task was considered to be completed.

The team has also practiced the collective ownership principle by sharing and collaborating on the project's code on GitHub. In addition, it was agreed with the customer on following the Java code convention and Android code convention in the development.

\paragraph{Design improvement and optimization}
The team has continuously followed this practice and also practiced it during design improvements in the development of prototypes.
 
\paragraph{Small releases}
The team has followed this practice by having sprints with a duration of two weeks, with a release for each sprint.


\paragraph{Simple design}
Although the team worked iteratively and in close collaboration with the customer, it has not always been easy to \emph{only} implement the basic functionality the customer requested. The initial assignment was very open for interpretation, \todo{and because it was the team's task to define the requirements specification, it was hard not to try to add functionality that we sincerely thought would improve the app, - very long sentence} \todo{even though that functionality might not had been marked as a high priority assignment.problem: vi har ikke satt prioritet på alle oppgavene..}

\paragraph{Sustainable pace}
The team has practiced this by having an estimation process where the amount of work hours were set to twenty hours a week.

\paragraph{On-site customer}
This principle does not apply to the team's project, as our customer comes from an external organization. However,\todo{ the team were co-located with our customer at a regular basis, usually once a week and hence got continuous feedback and acceptance testing.}\todo{henvise til iterativ testing i testing-kapittelet}

\paragraph{Test-first development}
The team has not followed this practice. Although it was initially thought that it would be a good idea to have a test-driven development approach towards the testing process, it was quickly realized that the team both lacked the experience and the time to follow this practice and therefore chose to drop it.


