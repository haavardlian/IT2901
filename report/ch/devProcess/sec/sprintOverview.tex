\section{Sprint overview}
The following sections try to summarize how the project progressed in a chronological order. They include the goals the team set for each sprint, and show if these goals were met.

\todo{add reference to backlog for each sprint}

\subsection{Kickoff}
An initial meeting was conducted not long after the team was assigned to this project. During this meeting it was discussed which times that were best suited to have weekly meetings. We also discussed and clarified what expectations we had to the project. All team members agreed to aim for the best grade possible.

\subsection{Sprint 1}
The goals set for this sprint were:\\\\
- to have a meeting with the customer,\\
- get a better grip of what the tasks ahead were,\\
- research about the task and \\
- assign all the team members roles.\\\\
After choosing the Scrum model as the project's framework, a rough project plan was made, and concepts for the app presented to the customer. The customer was very pleased with the teams ideas and concepts for the app, and approved the temporary specification. The team also started to set up the development environment by starting to write documentation in \LaTeX, setting up Android studio and choose Yodiz as the project management tool for work distribution and tracking.\todo{add reference to yodiz section}

\subsection{Sprint 2}
In this sprint the team focused on providing the customer with a working
prototype to further develop the requirements specification. Furthermore, a
working prototype might help the customer see how the team envisioned the app
and its functions. Lastly, work on the server app began to form the
foundation a framework of functions that the Android app could use.

The customer was happy with progress on the prototype, and provided a rich and detailed feedback on how the app should be improved for the next iteration.

\subsection{Sprint 3}
This sprint was dedicated to work on the customers wishes for the app. A customer meeting was held on the last day of last sprint. The meeting gave us a lot of feedback on what we were to focus on in further development. We experienced that it was hard to write sections of code without having a GUI design to follow. The team therefore decided to design prototypes before starting the implementation, resulting in detailed, digitized prototypes for each part of the app. The team also did some progress on the actual implementation.

During this sprint, some of  the team members got sick. This was also a period where other courses at NTNU had big exercise deliveries. The key issue was that the needed effort for the other subjects was underestimated. As this is hard to foresee, the team can only but allocate more time to make up for what was lost this sprint.

\subsection{Sprint 4}
This sprint, the team started implementing a prototype with functionality from the requirements specification. This prototyping included content for almost all of the tabs in the app. In addition to this, the team held a presentation for other course attendees, gave another group a peer evaluation on their preliminary report, and handed in our preliminary report for evaluation.

This sprint had two factors that made it into a productive, but also exhausting sprint for the team members: Because the previous sprint had not met expectations on completed work, the team tried to compensate for this by working overtime, exceeding the expected working hours by 30 percent. Naturally, the progress was uplifting, but we also got exhausted from all the extra work. It was decided to up the estimates on the next sprint since it became apparent that estimations so far had been too conservative. The team agreed it would be better to overestimate and add tasks to the sprint rather than underestimate and not finish the goals set in conjunction with the customer.

\subsection{Sprint 5}
After sprint 4, our app started to take form. This sprint consisted of further development from a prototype to something stable and functional, combined with a lot of important design choices. The customer expressed that he wanted to become even more involved with the development process by taking part in the sprint start meetings.
% where tasks were estimated, prioritized and allocated resources. 
He also gave us a lot of constructive feedback that lead to many major redesigns, particularly in the Android navigation menu. We were made aware of that navigation in what was the main menu at the time could be a challenge. The team laid out user stories according to the system specification, and after discussing each point, the team and customer agreed to sprint layouts for the remainder of the project.

This sprint had some goals that were not met. Data synchronization was not implemented and all dummy data were not yet replaced by data from the server based on user credentials. However, most of the apps major functionality was in place, and the team had a pretty clear image of what the finished app would look like and how it would behave.

\subsection{Sprint 6}
Sprint 6 was aimed at completing all functionality with the app to allow for user testing and rigorous testing and bug fixing. Completing the persistence layer was crucial to testing all the apps functionality. All tabs were to be completed at the end of this sprint. The only exception would be particularly tricky problems that required more resources that what were initially allocated.

After looking at the progress of sprint 6, the team decided it was time to dedicate some resources to the report in order to not fall behind. Sprint 5 and 6 had been devoted almost exclusively to development in order to meet deadlines set in cooperation with the customer. Even though generous estimates were made to ensure delivery, the team divided into two groups as an extra precaution. 

\subsection{Sprint 7}
The goals set for this sprint were:\\\\
- to finish development of the app and the server,\\
- run the test sets in the testing scheme,\\
- optimize code and fix bugs and\\
- acquire the customers approval for the finished product.\\\\
Bug fixes and optimizations made up for most of the development in sprint 7. The testing revealed performance issues and some major stability issues that had to be resolved before customer approval. Half of the team was working intensively on this while the other half had started focusing more on the report. A lot of work, especially in the chapter on further development had to be well under way before the start of sprint 8. After fixing the major issues discovered in testing, the team acquired the customers approval for the finished product on the 16th of May on the biweekly meeting.

\subsection{Sprint 8}
