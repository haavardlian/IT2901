\section{Sprint overview}


\subsection{Sprint 1}
Sprint 1 was the first actual sprint that the team had. The goals that were set for the sprint was to get a meeting with the customer, get a better grip of what the tasks ahead was, continuous research about the task and set all the team members roles.

The main focus of this sprint was the project management part. The scrum model was chosen as project methodology. A rough project plan was made, and concepts for the application was made.
The customer was very pleased with the teams ideas and concepts for the app, and approved of the temporary specification. The team began writing documentation in \LaTeX. To uphold the scrum model, the tool Yodiz was chosen. More about Yodiz can be found in section~\ref{sec:yodiz}.

The customer wanted the application to be developed at the Android platform. Android-studio became the tool of choice for this.

The team members deemed sprint 1 very successful

\subsection{Sprint 2}
In the second sprint the team focused on providing the customer with a working
prototype to further develop the requirements specification. Furthermore, a
working prototype might help the customer see how the team envisioned the app
and its functions. Lastly, work on the server application began to form the
foundation a framework of functions that the android app could use.

The customer was happy with progress on the prototype, and provided rich and detailed feedback on how the app should be improved for the next iteration. The sprint was deemed successful by the team members.

\subsection{Sprint 3}
This sprint was dedicated to work on the customers wishes for the application. A customer meeting had been done on the last day of last sprint. The meeting gave us a lot of good feedback about what to focus on. What the team experienced had experienced previously was that it was hard to work on writing the application for the parts that was not well prototyped.
The decision thus, was to focus on having detailed, well thought prototypes, that would satisfy the customers feedback. For the parts of the application that we don't have time to implement, there will be a textual description.

After this sprint the application has detailed prototypes for each part of the application. \todo[inline]{Må ha bilder av prototypene.} Some progress was also made on the actual implementation of the application.

There are two things to highlight in this sprint. The first is that the team has seen that much time is being spent discussing a lot. The discussion are of importance, but it drains too much of the time the team members has. As to this, it has been decided that there should be set a certain amount of time for the discussions. Should the discussions last longer than the time set, the project leader's task was to focus on ending the discussions, and instead perform a democratic vote based on the arguments that were presented. 

Another issue is that not enough time was spent. During this sprint, some team  members have been sick. This was also a period where other subjects had big exercise deliveries. The key issue was that the needed effort for the other subjects was underestimated. As this is hard to foresee, the team can only but allocate more time to get back what was lost this sprint.

\subsection{Sprint 4}
This sprint had two main goals. First of all the team started implementing a prototype with functionality from the requirements specification. This prototyping included content for almost all of the tabs in the application. In addition to this the team had a lot of work for the course IT2901. This included holding a presentation for the other groups taking the course, giving them peer evaluation on their preliminary report and handing in the teams preliminary report for evaluation.

This sprint had two factors that made it into a very productive, but very exhausting sprint for the team members. The previous sprint had not met expectations on completed work and the users stories in the current sprint had been underestimated. This lead to the team working overtime, exceeding the expected working hours by 30 percent. Naturally, the progress was uplifting, but the team was at the same time exhausted from all the extra work. It was decided to up the estimates on the next sprint since it became apparent that estimations so far had been too conservative. The team agreed it would be better to overestimate and add tasks to the sprint rather than underestimate and not finish the goals set in conjunction with the customer.

\subsection{Sprint 5}
After sprint 4, the application had started to take form. Sprint 5 consisted of further development from prototype to something stable and functional, combined with a lot of important design choices. The customer expressed that he wanted to become even more involved with the development process by taking part in the sprint start meetings where tasks were estimated, prioritized and allocated resources. The customer also gave a lot of constructive feedback that lead to many major redesigns, particularly in the menus. We were made aware of that navigation in what was the main menu at the time could be a challenge. The team laid out user stories according to the system specification, and after discussing each point, the team and customer agreed to sprint layouts for the remainder of the project.

The sprint had some goals that were not met. Data synchronization was not implemented and all dummy data was not yet replaced by data from the server based on user credentials. However, most of the applications major functionality was in place, and the team had a pretty clear image of what the finished application would look like and how it would behave.

\subsection{Sprint 6}
Sprint 6 was aimed at completing all functionality with the application to allow for user testing and rigorous testing and bug fixing. Completing the persistence layer was crucial to testing all the applications functionality. All tabs were to be completed at the end of this sprint. The only exception would be particularly tricky problems that required more resources that what were initially allocated.

After looking at the progress of sprint 6, the team decided it was time to dedicate some resources to the report in order to not fall behind. Sprint 5 and 6 had been devoted almost exclusively to development in order to meet deadlines set in cooperation with the customer. Even though generous estimates were made to ensure delivery, the team divided into two groups as an extra precaution. 

\subsection{Sprint 7}

\subsection{Sprint 8}
