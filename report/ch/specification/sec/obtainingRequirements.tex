\section{Obtaining customer requirements}
\label{sec:obtainingreq}

The customer did not provide an ordinary list of specifications at the beginning of the project. Instead, the customer provided the team with a general idea of what was wanted and a list of concepts that should be included in the application. It was up to the team to map out a preliminary list of features that could later be approved in conjunction with the customer. The customer wanted the team produce an application that is user friendly and simple, 
so that anyone can measure their energy consumption. This should be done preferably per device. The users should be able to share their energy consumption and -production with their friends on \gls{facebook}. 
The customer also wants the concept of \gls{gamification} to be brought into the application by allowing the users to compete with each other to save and/or produce the most energy. 
In order to measure the users' energy consumption per device, the team needed a hardware device. Optimally, the device transmits the data either via Wifi, Bluetooth, or another form of wireless communication protocol.

Due to time restrictions, the requirement of having hardware monitoring and control over each individual device was deemed optional, and only to be attempted if the team had an abundance of time. Naturally, this was not the case, and the team quickly agreed with the customer to drop these requirements from the specification. For further development the team agreed to do preliminary research on possible hardware solution provided by the customer. This was done to ease integration in the future if the customer wished to continue work on the application.

The team came up with a suggestion to requirements specifications that was discussed and decided in collaboration with the customer. The requirements final version of the requirements specification is described in detail in the next chapter.