\section{Elicitation of requirements}
\label{sec:obtainingreq}

The customer did not provide a distinct set of specifications when the project first started. Instead, the team was provided with a general idea and a list of concepts that the customer requested to be included in the application. It was up to the team to map out a preliminary list of features that could later be approved in conjunction with the customer. 

Some of these requirements were:
\begin{itemize}
\item The application should be user friendly and simple, so that anyone could measure their energy consumption. This was to be done preferably per device. 
\item The users should be able to share their energy consumption and -production with their friends on \gls{facebook}. 
\item The concept of \gls{gamification} to be brought into the application by allowing the users to compete with each other to save and/or produce the most energy. 
\end{itemize}

\noindent To measure the users' energy consumption per device, the team needed a hardware device. Optimally, the device transmits the data either via Wifi, Bluetooth, or another form of wireless communication protocol.

Due to time restrictions, the requirement of having hardware monitoring and control over each individual device was deemed optional, and only to be attempted if the team had an abundance of time. As the deadline for the final submission of the project approached, it became clear that this was not the case, and the team agreed with the customer to drop these requirements from the specification. 

However, the team agreed to do preliminary research on possible hardware solution provided by the customer. This was done to ease integration in future development if the customer were to continue to work on the application.
\todo[inline]{legg til en refereranse til denne researchen og further work}

In addition to the customer's own requirements, the team came up with some suggestions to requirements specifications that were discussed and added in collaboration with the customer. %Further details of the final version of the requirements specification are described in the subsequent sections.