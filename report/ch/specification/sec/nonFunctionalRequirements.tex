\section{Non-functional requirements}
\label{sec:nonFunctionalReq}
The non-functional requirements of a system are requirements that can not be directly translated into explicit functionality, but rather a set of criteria that may be used to evaluate the overall system. The testing of these requirements is described in chapter~\ref{sec:testingnonfunctionalrequirements}.

\subsubsection{NFR1: The app should be easy to use}
More generally, the user interface should be easy to use, also for people who are not accustomed to Android devices. In addition to following Android specifications, this will be achieved by keeping the interface as plain and intuitive as possible. 

\subsubsection{NFR2: The app's user interface should follow standard Android specifications for graphical user interfaces}
To ensure familiarity for users with previous experience with Android, the user interface should follow standard Android specifications for graphical user interfaces.
 
\begin{comment}

\subsubsection{NFR2: Installation guide and documentation}
The system should come with a comprehensive guide to using the system. This should include documentation for operation of the
app along with a guide to what the rest of the system operates. As the team will not be developing hardware for metering and 
aggregating data from homes, the main focus will be on the app and the team's central server. The documentation will contain an outline 
of the intended architecture for the rest of the system.


\begin{itemize}
\item The user interface should follow standard Android specifications. 
\item The applification should not need a lot of extra material to get started (Power devices, own server).
\end{itemize}

\end{comment}
