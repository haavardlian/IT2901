\section{Non-functional requirements}
\label{sec:nonFunctionalReq}
The non-functional requirements of a system are requirements that can not be directly translated into explicit functionality, but rather a set of criteria that may be used to evaluate the overall system. The testing of these requirements is described in chapter~\ref{sec:testingnonfunctionalrequirements}.

\subsubsection{NFR1: The app should be easy to use}
The user interface should be easy to use. It should not be necessary with extensive technical understanding or any domain knowledge. The interface should be as plain and intuitive as possible. 

\subsubsection{NFR2: The app should follow standard Android specifications}
To ensure familiarity to users with Android experience, the app should follow Android specifications. This is especially important in the graphical user interface to make sure it is easy for users to start using new apps.
 
\begin{comment}

\subsubsection{NFR2: Installation guide and documentation}
The system should come with a comprehensive guide to using the system. This should include documentation for operation of the
app along with a guide to what the rest of the system operates. As the team will not be developing hardware for metering and 
aggregating data from homes, the main focus will be on the app and the team's central server. The documentation will contain an outline 
of the intended architecture for the rest of the system.


\begin{itemize}
\item The user interface should follow standard Android specifications. 
\item The applification should not need a lot of extra material to get started (Power devices, own server).
\end{itemize}

\end{comment}
