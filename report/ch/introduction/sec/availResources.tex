\section{Available resources}
\label{sec:availResources}
The team had several available resources throughout the project, including work hours, our institute supervisor and the customer representative. To handle these resources, deadlines were set for important events, enabling the team to create a more satisfying product.

\subsection{Milestones}

The assignment and team was assigned January 20th. The initial customer meeting took place January 24th, where the assignment was presented and discussed. The final presentation of the app was held for the customer May 23rd at SINTEF. Table~\ref{tab:milestones} gives an overview of the important dates the team dealt with throughout the project. 

\begin{table}[H]
\centering
\rowcolors{1}{darkgray}{lightgray}
\begin{tabular}{|l|l|p{8cm}|}
\hline
\textbf{Deadlines} & \textbf{Name} & \textbf{Description}\\
07.02.14 & Sprint end 1 & First sprint is ended. \\
09.02.14& Report - preliminary version & Delivery of report outline for IT2901.\\
21.02.14& Sprint end 2 & Second sprint is ended. \\
07.03.14& Sprint end 3 & Third sprint is ended.\\
16.03.14& Report - mid-semester version &  Delivery of report draft for IT2901. \\
19.03.14 & Oral presentation of project & Present project for IT2901 students.\\
21.03.14& Sprint end 4 & Fourth sprint is ended.\\
21.03.14 & Feature freeze & No changes in the specification after this date.
\\
23.03.14 & Peer evaluation &  Evaluation of other IT2901 group's mid-semester report. \\
04.04.14& Sprint end 5 & Fifth sprint is ended.\\
02.05.14& Sprint end 6 & Sixth sprint is ended.\\
16.05.14& Sprint end 7 & Seventh sprint is ended.\\
16.05.14& Code freeze & All further development is stopped.\\
23.05.14& Product delivery to customer & Presentation for SINTEF.\\ 
30.05.14& Sprint end 8 & Eight sprint is ended.\\
30.05.14 & Report - final version & The finalized report was delivered.\\\hline
\end{tabular}
\caption{Milestones and important events}
\label{tab:milestones}
\end{table}

\subsection{Available hours}

Each sprint had a duration of two weeks, excluding the period from April 4th to April 21th, when the entire group left for a school field trip to China. To compensate for the days missed because of the school field trip, it was decided to increase the amount of work hours from 20 to 25 as of sprint 2 up to and including sprint 8. Table~\ref{tab:availHours} lists the project's available hours and is based on all six team members spending at least the required hours on the project every week.

\begin{table}[H]
\centering
\rowcolors{1}{darkgray}{lightgray}
\begin{tabular}{|l|l|l|l|}
\hline
\textbf{Period} & \textbf{Dates} & \textbf{Days} & \textbf{Hours}\\
Sprint 1& January 27 - February 07 & 10  & 240 \\
Sprint 2 & February 10 - February 21 &10  & 300 \\
Sprint 3 & February 24 - March 07 &10 & 300 \\
Sprint 4 & March 10 - March 21 &10  &300 \\
Sprint 5 & March 24 - April 04 &10&  300 \\
Sprint 6 & April 21 - May 02 &10  &300 \\
Sprint 7 & May 05 - May 16 &10  &300 \\
Sprint 8 & May 19 - May 30 &10  &300 \\
\textbf{Total}&& \textbf{80}&  \textbf{2340}\\\hline
\end{tabular}
\caption{Available hours}
\label{tab:availHours}
\end{table}


\subsection{Supervisor from the Department of Computer and Information Science}
The team's supervisor was Alfredo Perez Fernandez. He is a PhD student at NTNU in the Department of Computer and Information Science. He may be contacted by e-mail at perezfer@idi.ntnu.no.

\subsection{Customer representative at SINTEF}
The team's customer representative at SINTEF was Babak Farshchian. He is an adjunct associate professor at NTNU and a researcher at SINTEF. He may be contacted by e-mail at babak.farshchian@sintef.no.
