\newpage
\section{Available resources}

During the project scope, the team had several available resources, including work hours, our institute supervisor and the customer representative. In order to cope with these resources, deadlines were set for important events and milestones so that the team would be better able to create a satisfying product.


\subsection{Available hours}

The assignment and team members were assigned on January 20. The initial meeting with the customer took place January 24, where the assignment was presented and discussed. Oral presentation of the project was performed March 19 and the final deadline for submission of the project was set to May 30. The team set the deadline for new specifications from the customer that would result in major changes to the software, to March 21, due to time restrictions.

Each sprint had a duration of two weeks, excluding the period from April 4 to April 21, when the entire group left for a school field trip to China. To compensate for the days the team missed because of the school field trip mentioned above, the team decided to increase the amount of work hours from 20 hours to 25 hours as of sprint 2 to and including sprint 8. Table~\ref{tab:availHours} lists the project's available hours and is based on that all team members spends at least twenty hours on the project every week.



\begin{table}[H]
\centering
\rowcolors{1}{darkgray}{lightgray}
\begin{tabular}{|l|l|l|l|}
\hline
\textbf{Period} & \textbf{Dates} & \textbf{Days} & \textbf{Hours}\\
Sprint 1& January 27 - February 07 & 10  & 240 \\
Sprint 2 & February 10 - February 21 &10  & 300 \\
Sprint 3 & February 24 - March 07 &10 & 300 \\
Sprint 4 & March 10 - March 21 &10  &300 \\
Sprint 5 & March 24 - April 04 &10&  300 \\
Sprint 6 & April 21 - May 02 &10  &300 \\
Sprint 7 & May 05 - May 16 &10  &300 \\
Sprint 8 & May 19 - May 30 &10  &300 \\
\textbf{Total}&& \textbf{80}&  \textbf{2340}\\\hline
\end{tabular}
\caption{Available hours}
\label{tab:availHours}
\end{table}


\subsection{Milestones}
\todo[inline]{needs description and possibly more dates. Also needs double checking of that the dates are correct.}
\begin{table}[H]
\centering
\rowcolors{1}{darkgray}{lightgray}
\begin{tabular}{|l|p{6.7cm}|p{6.5cm}|}
\hline
\textbf{Deadlines} & \textbf{Name} & \textbf{Description}\\
07.02.14 & Sprint end 1 & First sprint is ended. \\
09.02.14& Project report - preliminary version & \\
21.02.14& Sprint end 2 & Second sprint is ended. \\
07.03.14& Sprint end 3 & Third sprint is ended.\\
16.03.14& Project report - mid-semester version &   \\
19.03.14 & Oral presentation of the project&\\
21.03.14& Sprint end 4 & Fourth sprint is ended.\\
21.03.14 & Feature freeze & After this date, it was not possible to change the specifications if it resulted in major changes in the software.\\
23.03.14 & Peer evaluation &  \\
04.04.14& Sprint end 5 & Fifth sprint is ended.\\
02.05.14& Sprint end 6 & Sixth sprint is ended.\\
19.05.14& Product delivery to customer. Når skal vi lever vs framføre? &\\ 
30.05.14 & Project report - final version & The finalized report was delivered.\\\hline
\end{tabular}
\caption{Milestones and important events}
\label{tab:milestones}
\end{table}


\subsection{Supervisor from the Department of Computer and Information Science}
The team's supervisor was Alfredo Perez Fernandez. He is a PhD student at NTNU in the Department of Computer and Information Science~\cite{idi}. He may be contacted by e-mail at perezfer@idi.ntnu.no.

\subsection{Customer representative at SINTEF}
The team's customer representative at SINTEF was Babak Farshchian. He is an adjunct associate professor at NTNU and a researcher at SINTEF ICT~\cite{sintefict}. He may be contacted by e-mail at babak.farshchian@sintef.no.
