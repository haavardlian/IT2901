\subsection{Performance}
A seamless user experience is a big part of usability. Immediate response in the GUI was therefore a high priority during development. This proved a sizeable challenge as all data in the app is synchronized with the server. 
The sheer amount of data also made the graph library render the graphs slower than satisfactory. This prompted a lot of query optimization and tweaking 
in both the Android app and the server. 
The customer was initially sceptical to using a dedicated server to store data as this adds the responsibility of keeping the server running and performing for the app to be usable. Unfortunately, no peer-to-peer or similar solutions satisfied the teams requirements to data persistence and speed.

\subsection{Maintainability}
The server does not require much maintenance to be kept running at peek performance. Much of the server's performance depends on the hardware and Internet connection. 

\subsection{Scalability}
The software is easily scaled for the hardware by changing the configuration.  To make room for a growing user base and more powerful hardware, this can also be done for the server, by changing the allowed number of simultaneous requests and connections.


\subsection{Modularity}
\label{sec:modularity}
As the final development architecture deviated from the architecture that was envisioned for a perfect product, a modular design was critical to allow for easy modifications and 
extensions of the software in the future. 

The server has an inert  modularity in the fact that is based on stateless connections that transfer JSON objects. Adding hardware that can control and monitor devices in the users 
home can easily interface to the app by storing data on the server. By either using predefined functions, expanding upon them or by writing new functions hardware devices or a home data aggregation server can transfer data to the main server. No changes would be necessary on the client side to add this functionality. 
