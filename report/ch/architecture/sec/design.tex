\section{Architectural design decisions}
\label{sec:arch_design}
Selecting the architecture is an important part of any software development project. To create the best possible solution, a preliminary study was performed and a plan on how the different components would interact, was created. 

The team members had little to no experience with developing a server according to the chosen design. Development and design was therefore done interchangeably and iteratively in the early stages of the project. Changes in the architectural design were made because many of the requirements in the first draft were discarded due to time constraints.

The server should be simple and fast, which is ideal for the stateless design that was ultimately chosen. 

The customer was initially skeptical to use a dedicated server to store data as this adds a responsibility of uptime and maintenance. The customer suggested alternative solutions, like Dropbox and Google Drive.
These were incompatible with several of the functional requirements, including tips(FR3), and comparison functionality(FR5.4). In addition, this would limit furter development. 

When selecting an architecture, the following properties were evaluated. 

\subsection{Performance}
A responsive user interface had a high priority during development. This proved to be a challenge, as all the app data on the device required synchronization with the server. The amount of data also made the graph library render slower than desired. To solve this, query optimization and tweaking on both the Android app and the server was necessary.

\subsection{Maintainability}
The server does not require much maintenance to run at peek performance. Much of the server's performance depends on the hardware and Internet connectivity. If the user base grows, or additional heavy server functionality is added, the hardware can be upgraded to keep up.

\subsection{Modularity}
\label{sec:modularity}
As the final development architecture deviated from the architecture that was envisioned for an optimal product, it was critical to have a modular design in order to allow for easy modifications and extensions of the software in the future. 

The server has an inherent modularity in the fact that it is based on stateless connections, making it independent from the app. Functionality that monitors and controls devices in the user's home can easily be added, and interfaced with the existing solution without major changes. No changes would be necessary on the client side to add this functionality.
