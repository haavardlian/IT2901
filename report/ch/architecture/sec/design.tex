\section{Architecture design decisions}
Choosing an architecture is probably the most influential decision in any software development project. The pre-study conducted earlier gave the team a solid base of knowledge to base architectural 
decisions on. Even before the pre-study the team had some ideas on what parts the system should consist of and how they would interact, but getting an overview of how other systems were built was 
very helpful. As mentioned in the Requirements chapter, the original requirements specification was discarded due to time constraints. However, the architecture was built with the original architecture in mind, so it can be easily extended to meet the original requirements.

The design of the entire system started very abstract, but quickly developed into a detailed plan of how the different components would interact. As few of the team members had much 
experience with developing a server according to the design, development and design were done interchangeably and iteratively in the early stages of the project. This is also true for the 
android development process. As new features or problems were discovered in early development, the architecture design had to undergo updates regularly. A key concern for the team was that the 
server should be simple and fast. This is ideal for the stateless design that was ultimately chosen. 