\section{Architectural design decisions}
\label{sec:arch_design}
Selecting the architecture is an important part of any software development project. To create the best possible solution, a preliminary study was performed and a plan on how the different components would interact, was created. 

The team members had little to no experience with developing a server according to the chosen design. Development and design was therefore done interchangeably and iteratively in the early stages of the project. Changes in the architectural design were made because many of the requirements in the first draft were discarded due to time constraints. 

The customer was initially skeptical to use a dedicated server to store data as this adds a responsibility of up-time and maintenance. The customer suggested alternative solutions, like Dropbox and Google Drive.
These were incompatible with several of the functional requirements, including tips (FR3), and comparison functionality (FR5.4). In addition, this would limit further development. 

To cover all the functional requirements, storing the data on a central server was necessary. This resulted in a client-server architecture. 