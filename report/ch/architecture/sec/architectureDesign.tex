\section{Designing the architecture}
Choosing an architecture is probably the most influential decision in any software development project. The pre-study conducted earlier gave the team a solid base of knowledge to base architectural  decisions on. Even before the pre-study the team had some ideas on what parts the system should consist of and how they would interact, but getting an overview of how other systems were built was very helpful. 

\subsection{Performance and maintainability}
A seamless user experience is a big part of usability. Immediate response in the GUI was therefore a priority during development. This proved a sizeable challenge as all data in the app is synchronized with the server. The sheer amount of data also made the graph library render the graphs slower than satisfactory. This prompted a lot of query optimization and tweaking in both the android app and the server. The customer was initially sceptical to using a dedicated server to store data as this adds the responsibility of keeping the server running and performing for the app to be usable. Unfortunately, no peer-to-peer or similar solutions satisfied the teams requirements to data persistence and speed.

The server does not require much maintenance to keep running at peek performance. Much of the performance of the server depends on the hardware and internet connection. The software itself is easily scaled for the hardware by changing the configuration. By changing the allowed number of simultaneous requests and connections, the server can easily be scaled for a growing user based and more powerful hardware.

\subsection{Modularity}
As the final development architecture deviated from the architecture that was envisioned for a perfect product, a modular design was critical to allow for easy modifications and extensions of the software in the future. 