\section{Risk analysis}
\label{sec:risk}
As part of our project planning, we outlined potential risks to the progress of our project. A risk is defined as an unwanted event that has a negative affect on the process. We acknowledged the possibility of challenges and problems that might arise during the project, such as technical problems, human errors or personal problems. These are risks that might apply to both individuals and the entire team. Effects of these problems include delays, conflicts and anything that might slow down the progress, and ultimately lead to failure to meet deadlines. 

The risk elements are sorted by their importance. Importance is calculated with two factors in mind; the calculated probability of the event actually occurring and the effect the event will have on the project. 

In the table given below, we analyze the elements we consider risks to our project. Likelihood (L) and effect (E) is measured on a scale from 1 to 9. For likelihood, a 9 is very likely and a 1 is very unlikely. For effect, a 9 is devastating and a 1 is very little effect. The table is sorted from high to low importance (I). Importance is the product of probability and effect. 
\newpage

\rowcolors{0}{lightgray}{darkgray}
\setlength{\tabcolsep}{6pt}
\begin{longtable}{|p{0.5cm}!{\vrule width -1pt}p{3.6cm}!{\vrule width -1pt}l!{\vrule width -1pt}l!{\vrule width -1pt}l!{\vrule width -1pt}p{4.4cm}!{\vrule width -1pt}p{4.4cm}|}


\hline\multicolumn{1}{|l!{\vrule width -1pt}}{\cellcolor{darkgray}\textbf{\#}} &\multicolumn{1}{l!{\vrule width -1pt}}{\cellcolor{darkgray}\textbf{Description}} &  \multicolumn{1}{c!{\vrule width -1pt}}{ \cellcolor{darkgray}\textbf{L}} & \multicolumn{1}{c!{\vrule width -1pt}}{\cellcolor{darkgray}\textbf{E}} & \multicolumn{1}{c!{\vrule width -1pt}}{\cellcolor{darkgray}\textbf{I}} & \multicolumn{1}{l!{\vrule width -1pt}}{\cellcolor{darkgray}\textbf{Preventive actions}} & \multicolumn{1}{l|}{\cellcolor{darkgray}\textbf{Remedial actions}}\\ 
\endfirsthead


\multicolumn{7}{c}%
{{\bfseries \tablename\ \thetable{} -- continued from previous page}} \\
\hline\multicolumn{1}{|l!{\vrule width -1pt}}{\cellcolor{darkgray}\textbf{\#}} &\multicolumn{1}{l!{\vrule width -1pt}}{\cellcolor{darkgray}\textbf{Description}} &  \multicolumn{1}{c!{\vrule width -1pt}}{ \cellcolor{darkgray}\textbf{L}} & \multicolumn{1}{c!{\vrule width -1pt}}{\cellcolor{darkgray}\textbf{E}} & \multicolumn{1}{c!{\vrule width -1pt}}{\cellcolor{darkgray}\textbf{I}} & \multicolumn{1}{l!{\vrule width -1pt}}{\cellcolor{darkgray}\textbf{Preventive actions}} & \multicolumn{1}{l|}{\cellcolor{darkgray}\textbf{Remedial actions}}\\ \hline 
\endhead

\hline \multicolumn{7}{|l|}{{Continued on next page}} \\ \hline
\endfoot
 
\endlastfoot

\hline
1&Underestimation of workload & 9 & 8 & 72 & Continuously revise workload and how much time is left, and prioritize & Reestimate continuously\\
2& Issues with software or tools & 9 & 5 & 45 & Choose software the team members are familiar with & Hold workshops and view tutorials\\
3&Customer has not fulfilled his obligations & 8 & 5 & 40 & Keep customer updated and maintain continuous communication. Set final deadlines for feedback. & Contact supervisor.\\
4&Illness & 9 & 4 & 36 & Multiple team members work on the same task& Allocate sick member's task to remaining members\\
5& Unbalanced workload & 5 & 7 & 35 & Coordinate with entire team and log hours & Reallocate tasks\\
6&Unproductive hours caused by external disruptions & 7 & 5 & 35 & & Find another place to work. Work extra hours to make up for lost time. \\
7&Team member unavailable & 9 & 3 & 27 & Inform each other of dates we know we will be unavailable. Good infrastructure for communication and progress reports & Keep in touch with unavailable team member or redistribute tasks\\
%Lars and Beate would possibly have to change to IntelliJ, so this might disappear. Android -> IntelliJ. Back-end->Eclipse/IntelliJ.
%&Eclipse and IntelliJ is incompatible & 7 & 3 & 21 & Evaluate whether use of different IDEs have any negative impact on project & Decide to choose only one of them. \\
8&External services unavailable & 4 & 7 & 21 & Project not fully dependent of external services. Server is easy to deploy. & Must find new external service to replace the one(s) becoming unavailable. Deploy elsewhere.\\
9& Spending more time than estimated on discussions & 7 & 3& 21& Include a buffer in our estimations. & Project manager decides whether subject is worth the time and if it shall be rescheduled.\\
10&Data loss & 2 & 9 & 18 & Use version control system &Restore data from previous versions\\
11&Customer unavailable & 3 & 6 & 18 & Keep regular contact with customer & Discuss problem within team and contact supervisor\\
12&Communication failure & 7 & 2 & 14 & Make e-mail-list and exchange contact information. Be explicit. & Check e-mail and phones multiple times a day.\\
13&Customer requirements exceed predicted amount of time& 4 & 3 & 12 & Continuously revise the workload and prioritize. Set a date for last major changes. & Politely explain that there is not enough time for the changes he suggests. Drop low-priority tasks\\
14&Supervisor unavailable & 3 & 4 & 12 & Keep regular contact with supervisor & Discuss problem internally and contact department.\\
%13&Conflict in group & 5 & 2 & 10 & Have in mind that people often misunderstand each other and have an open dialog to solve problem. & Contact supervisor for help.\\
15&Team member has not completed given assignments & 1 & 7 & 7 & Keep track of the effort and the hour's log public & Contact supervisor and redistribute tasks \\\hline
\caption{Risk analysis table}
\end{longtable}
\rowcolors{0}{}{}

\subsection{Description of risks}
\subsubsection{Underestimation of workload}
\subsubsection{Issues with software or tools}
\subsubsection{Customer does not fulfil obligations}
\subsubsection{Illness}
\subsubsection{Unbalanced workload}
\subsubsection{Loss in productivity caused by external factors}
\subsubsection{Team member unavailable}
\subsubsection{External services unavailable}
\subsubsection{Spending more time than estimated on discussions}
\subsubsection{Data loss}
\subsubsection{Customer unavailable}
\subsubsection{Communication failure}
\subsubsection{Customer requirements exceed predicted amount of time}
\subsubsection{Supervisor unavailable}
\subsubsection{Team member has not completed given assignments}


\subsection{Changes during the project}
\subsubsection{Removed risks}
\paragraph{Eclipse and IntelliJ are incompatible}
In the early phases of the project the team discussed what IDE should be used for development. As some members expressed that they wanted to continue using Eclipse instead of switching to IntelliJ, the team outlined compatability problems as a possible risk. When said team members later decided to use IntelliJ, this risk was removed.

\paragraph{Different skills and background}
Initially the team considered this a risk due to the amount of work that was due. This view was quickly repealed after a short discussion where the team came to the conclusion that diversity in skill sets was a big benefit.

\paragraph{Team member busy}
Initially the team noted both "team member busy" and "team member unavailable" as risks. The reasoning behind this was that these risks had different causes. The team quickly realized that the preventative measures, remedial measures and consequences where the same for these two risks, and therefore the former was removed. 