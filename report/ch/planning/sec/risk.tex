\section{Risk analysis}
\todo[inline]{Hele denne seksjonen bør revurderes: Den er veldig lang, både selve risikoanalysetabellen og beskrivelsene. Bør kanskje bare forklare deler av tabellen i en kortere tekst? Nå har vi over tre sider til disse beskrivelsene. That number is too damn high!}

\label{sec:risk}
As part of our project planning, we outlined potential risks to the progress of our project. A risk is defined as an unwanted event that has a negative affect on the process. We acknowledged the possibility of challenges and problems that might arise during the project, such as technical problems, human errors or personal problems. These are risks that might apply to both individuals and the entire team. Effects of these problems include delays, conflicts and anything that might slow down the progress, and ultimately lead to failure to meet deadlines. 

The risk elements are sorted by their importance. Importance is calculated with two factors in mind; the calculated probability of the event actually occurring and the effect the event will have on the project. 

In the table given below, we analyse the elements we consider risks to our project. Likelihood (L) and effect (E) is measured on a scale from 1 to 9. For likelihood, a 9 is very likely and a 1 is very unlikely. For effect, a 9 is devastating and a 1 is very little effect. The table is sorted from high to low importance (I). Importance is the product of probability and effect. 

\rowcolors{0}{lightgray}{darkgray}
\setlength{\tabcolsep}{6pt}
\begin{longtable}{|p{3.6cm}!{\vrule width -1pt}l!{\vrule width -1pt}l!{\vrule width -1pt}l!{\vrule width -1pt}p{4.8cm}!{\vrule width -1pt}p{4.8cm}|}

\hline
\multicolumn{1}{|l!{\vrule width -1pt}}{\cellcolor{darkgray}\textbf{Description}} &  \multicolumn{1}{c!{\vrule width -1pt}}{ \cellcolor{darkgray}\textbf{L}} & \multicolumn{1}{c!{\vrule width -1pt}}{\cellcolor{darkgray}\textbf{E}} & \multicolumn{1}{c!{\vrule width -1pt}}{\cellcolor{darkgray}\textbf{I}} & \multicolumn{1}{l!{\vrule width -1pt}}{\cellcolor{darkgray}\textbf{Preventive actions}} & \multicolumn{1}{l|}{\cellcolor{darkgray}\textbf{Remedial actions}}\\ 
\endfirsthead


\multicolumn{6}{c}%
{{\bfseries \tablename\ \thetable{} -- continued from previous page}} \\\hline
\multicolumn{1}{|l!{\vrule width -1pt}}{\cellcolor{darkgray}\textbf{Description}} &  \multicolumn{1}{c!{\vrule width -1pt}}{ \cellcolor{darkgray}\textbf{L}} & \multicolumn{1}{c!{\vrule width -1pt}}{\cellcolor{darkgray}\textbf{E}} & \multicolumn{1}{c!{\vrule width -1pt}}{\cellcolor{darkgray}\textbf{I}} & \multicolumn{1}{l!{\vrule width -1pt}}{\cellcolor{darkgray}\textbf{Preventive actions}} & \multicolumn{1}{l|}{\cellcolor{darkgray}\textbf{Remedial actions}}\\ \hline 
\endhead

\hline \multicolumn{6}{|l|}{{Continued on next page}} \\ \hline
\endfoot
 
\endlastfoot

\hline
Underestimation of workload & 9 & 8 & 72 & Continuously revise workload and time left, and prioritize & Reestimate continuously\\
Issues with software or tools & 9 & 5 & 45 & Choose software the team members are familiar with & Hold workshops and view tutorials\\
Customer does not fulfill obligations & 8 & 5 & 40 & Keep customer updated and maintain continuous communication. Set final deadlines for feedback. & Contact supervisor.\\
Illness & 9 & 4 & 36 & Multiple team members work on the same task& Allocate sick member's task to remaining members\\
Unbalanced workload & 5 & 7 & 35 & Coordinate with entire team and log hours & Reallocate tasks\\
Unproductive hours caused by external disruptions & 7 & 5 & 35 & & Find another place to work. Work extra hours to make up for lost time. \\
Team member unavailable & 9 & 3 & 27 & Inform each other of dates we will be unavailable. Good infrastructure for communication and progress reports & Keep in touch with unavailable team member or redistribute tasks\\
%Lars and Beate would possibly have to change to IntelliJ, so this might disappear. Android -> IntelliJ. Back-end->Eclipse/IntelliJ.
%&Eclipse and IntelliJ is incompatible & 7 & 3 & 21 & Evaluate whether use of different IDEs have any negative impact on project & Decide to choose only one of them. \\
External services unavailable & 4 & 7 & 21 & Project not fully dependent of external services. Server is easy to deploy. & Must find new external service to replace the one(s) becoming unavailable. Deploy elsewhere.\\
Spending more time than estimated on discussions & 7 & 3& 21& Include a buffer in our estimations. & Team leader decides if subject is not worth the time or should be rescheduled.\\
Data loss & 2 & 9 & 18 & Use version control system &Restore data from previous versions\\
Customer unavailable & 3 & 6 & 18 & Keep regular contact with customer & Discuss problem within team and contact supervisor\\
Communication failure & 7 & 2 & 14 & Make e-mail-list and exchange contact information. %Be explicit.
 & Check e-mail and phones multiple times a day.\\
Customer requirements exceed project time scope& 4 & 3 & 12 & Continuously revise the workload and prioritize. Set a date for last major changes. & Politely explain that there is not enough time for the changes he suggests. Drop low-priority tasks.\\
Supervisor unavailable & 3 & 4 & 12 & Keep regular contact with supervisor & Discuss problem internally and contact department.\\
%13&Conflict in group & 5 & 2 & 10 & Have in mind that people often misunderstand each other and have an open dialog to solve problem. & Contact supervisor for help.\\
Team member did not complete given assignments & 1 & 7 & 7 & Keep track of the effort and the hour's log public & Contact supervisor and redistribute tasks \\\hline
\caption{Risk analysis table}
\end{longtable}
\rowcolors{0}{}{}

\subsection{Description of risks}
The following sections give a short but explanatory description of all the risks the team included as potential threats to progress of the project.

\subsubsection{Underestimation of workload}
A prevalent risk to any project is failure to acknowledge the cost of tasks in the project and the improper allocation of resources that follows such an error. This can also lead to a false sense of what a team can achieve in a given time frame.

\subsubsection{Issues with software or tools}
Any team of software developers usually rely on a set of tools and software to help them develop and document a system. Unavailability or improper use of such tools can lead to setbacks in progress.

\subsubsection{Customer does not fulfill obligations}
A developer team often relies on information from the customer to progress their work. Failure to meet request or follow up on agreements can lead to the teams work flow being staggered.

This risk was added after having some slight trouble with communication and reliability with the customer. Early on, the team did not think this to be a risk that having a huge impact on the project progress, but acknowledged the possibility that bigger issues might arise in the future as a result of this risk.

\subsubsection{Illness}
Illness amongst team members can result in work being left undone. Dependant on the duration and severity, the consequences illness can be negligible or devastating. 

\subsubsection{Unbalanced workload}
Failure to estimate the cost of tasks, or failure to allocate them wisely, can leave some team members grasping for more work while others are bogged down with too heavy tasks.

\subsubsection{Loss in productivity caused by external factors}
Interruptions in meetings, bad air quality or construction work are some factors that can change the mood and productivity of team members. Such disturbances siphon time from the set working hours, resulting in a drop in productivity. 

\subsubsection{Team member unavailable}
Planned or unplanned leaves of absence can leave a team under staffed, and possibly lacking the required resources to complete tasks.

\subsubsection{External services unavailable}
APIs, web tools and other tools that a team utilize can become unavailable. Even though such cases are usually rare, loss of backward compatibility or functionality can be devastating for a system that depends on these services.

\subsubsection{Spending more time than estimated on discussions}
With a team of several opinionated individuals, discussion concerning the project have to possibility to get out of hand and take up more time than expected.

\subsubsection{Data loss}
Hardware does not last forever. Hard drives or data corruption can lead to loss of data and usually results in a set back for the individual in question.

\subsubsection{Customer unavailable}
Planned or unplanned unavailability from the customer can potentially lead to the progress of a project being halted pending input that could be crucial for continuation of the work flow.
As the team had not worked with a real customer before, this risk was overlooked in the first iteration.


\subsubsection{Communication failure}
Failure to communicate can happen in all subsets of parties involved in a project. Unclear or insufficient communication can lead to misunderstanding between team members, the team and the customer or between the team and the supervisor.

\subsubsection{Customer requirements exceed predicted amount of time}
A customers idea of what is possible in a given time frame might differ from what the team feels is possible. Inflated specifications of a system can lead to the team being unable to complete entire parts of the system or lowering the quality of the end product.

\subsubsection{Supervisor unavailable}
Planned or unplanned unavailability from the customer can leave a team without proper feedback and guidance which may or may not be influential to the quality of the project.

\subsubsection{Team member has not completed given assignments}
A team member that does not follow up on responsibilities might leave work undone. Depending on the grade of negligence, consequences can be negligible to devastating.

\subsubsection{External services unavailable}
After discussing some recent downtime in popular services with the supervisor, he pointed out, and the team became aware of the risk of using internet services like GitHub and Google Drive. Even though such services have a very good reliability record, the team calculated the risk and added it to the list.

\subsection{Changes during the project}
The team ran a continual re-evaluation of risks throughout the project. Some risks were changed due to experiences with some of the risks during the project, while others were changed purely on a theoretical basis. The teams own experience and knowledge combined with feedback from other experienced third parties led to the following changes in the risks.

\subsubsection{Too much to do}
This risk appeared in the preliminary report that the team produced early in the process. After feedback for the supervisor the team agreed that this risk was to vague. It was changed to "Customer requirements exceeds predicted amount of time".

\subsubsection{Short and long term illness}
These two risks were initially defined as two different risks. After realizing that preventative measure, remedial measure and consequences where identical, it was merged to the risk "Illness".

\subsubsection{Team member drops out}
This risk was changed several times during the project. The first change rephrased it as "Team member unwilling to participate" as the probability of dropping out was considered to be next to zero. The team decided to change it again to have a more formal and objective wording. Finally this risk was coined as "Team member has not completed given assignments"

\subsubsection{Eclipse and IntelliJ are incompatible}
In the early phases of the project the team discussed what IDE should be used for development. As some members expressed that they wanted to continue using Eclipse instead of switching to IntelliJ, the team outlined compatibility problems as a possible risk. When said team members later decided to use IntelliJ, this risk was removed.

\subsubsection{Different skills and background}
Initially the team considered this a risk due to the amount of work that was due. This view was quickly repealed after a short discussion where the team came to the conclusion that diversity in skill sets was a big benefit.

\todo[inline]{\textbf{usikker på om denne trenger å være med:}
Team member busy
Initially the team noted both ''Team member busy'' and ''Team member unavailable'' as risks. The reasoning behind this was that these risks had different causes. The team quickly realized that the preventative measures, remedial measures and consequences where the same for these two risks, and therefore the former was removed. }
