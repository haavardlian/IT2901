\section{Domain knowledge}
The project required research on two very different areas before the team could start with the design and production of the app that was to be developed. The customer wanted focus on raising awareness around energy consumption. To develop an app for this, a deeper insight was needed into how different factors effect a households  power consumption. Electrical power providers were contacted so that the team could learn what they knew about consumption and their tips to conserve energy. As the app's target group was private households, this was also the main focus of research. The feedback from the companies supplemented by Internet research gave the team a good understanding of what the most common issues were, and what people could do to fix them. The app was to provide users with these tips and the rating they had received from other users. 

Another field that had to be researched was potential hardware that the team could use to measure consumption in real time, and also control devices. Before the pre study was carried out, the customer discussed features that would set the app aside from the numerous similar ones already on the market. It was agreed that allowing the app to control individual devices like light bulbs and heaters, at the same time as consumption was automatically measured, would set the app in an unique segment of the market. Although the team later had to drop the functionality allowing for automatic control and measurement due to time constraints, quite a bit of work was laid down in this area. The team concluded that the only reasonable architecture for a fully automatic measurement would include a small server, running on a Raspberry Pi, and one measurement device with wireless transmitter connected to anything the user wanted to control and monitor. These particular devices that allow for wireless transmission proved to be hard to find as they have not achieved much success on the market. Much of this is due to the lack of standards in transmission between devices and servers. The team found that many of the existing apps would have a device that were desired, but that it would be proprietary technology. The customer had previously requested that the devices should be cheap and preferably based on open source technology. As mentioned, all these challenges combined with time constraints lead the team, in agreement with the customer, to develop the app without supporting this. Instead the app was to be developed in way to would allow easy integration of this functionality at a later point in time. More details on this can be found in chapter~\ref{sec:further}, along with more about how the team envisions further development. 