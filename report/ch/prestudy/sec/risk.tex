\section{Risk analysis}

\label{sec:risk}
As part of our project planning, we outlined potential risks to the progress of our project. A risk is defined as an unwanted event that has a negative affect on the process. We acknowledged the possibility of challenges and problems that might arise during the project, such as technical problems, human errors or personal problems. These are risks that might apply to both individuals and the entire team. Effects of these problems include delays, conflicts and anything that might slow down the progress, and ultimately lead to failure to meet deadlines. 

The risk elements are sorted by their importance. Importance is calculated with two factors in mind; the calculated probability of the event actually occurring and the effect the event will have on the project. 

In the table given below, we analyse the elements we consider risks to our project. Likelihood (L) and effect (E) is measured on a scale from 1 to 9. For likelihood, a 9 is very likely and a 1 is very unlikely. For effect, a 9 is devastating and a 1 is very little effect. The table is sorted from high to low importance (I). Importance is the product of probability and effect. 

\rowcolors{0}{lightgray}{darkgray}
\setlength{\tabcolsep}{6pt}
\begin{longtable}{|p{3.6cm}!{\vrule width -1pt}l!{\vrule width -1pt}l!{\vrule width -1pt}l!{\vrule width -1pt}p{4.8cm}!{\vrule width -1pt}p{4.8cm}|}

\hline
\multicolumn{1}{|l!{\vrule width -1pt}}{\cellcolor{darkgray}\textbf{Description}} &  \multicolumn{1}{c!{\vrule width -1pt}}{ \cellcolor{darkgray}\textbf{L}} & \multicolumn{1}{c!{\vrule width -1pt}}{\cellcolor{darkgray}\textbf{E}} & \multicolumn{1}{c!{\vrule width -1pt}}{\cellcolor{darkgray}\textbf{I}} & \multicolumn{1}{l!{\vrule width -1pt}}{\cellcolor{darkgray}\textbf{Preventive actions}} & \multicolumn{1}{l|}{\cellcolor{darkgray}\textbf{Remedial actions}}\\ 
\endfirsthead


\multicolumn{6}{c}%
{{\bfseries \tablename\ \thetable{} -- continued from previous page}} \\\hline
\multicolumn{1}{|l!{\vrule width -1pt}}{\cellcolor{darkgray}\textbf{Description}} &  \multicolumn{1}{c!{\vrule width -1pt}}{ \cellcolor{darkgray}\textbf{L}} & \multicolumn{1}{c!{\vrule width -1pt}}{\cellcolor{darkgray}\textbf{E}} & \multicolumn{1}{c!{\vrule width -1pt}}{\cellcolor{darkgray}\textbf{I}} & \multicolumn{1}{l!{\vrule width -1pt}}{\cellcolor{darkgray}\textbf{Preventive actions}} & \multicolumn{1}{l|}{\cellcolor{darkgray}\textbf{Remedial actions}}\\ \hline 
\endhead

\hline \multicolumn{6}{|l|}{{Continued on next page}} \\ \hline
\endfoot
 
\endlastfoot

\hline
Underestimation of workload & 9 & 8 & 72 & Continuously revise workload and time left, and prioritize & Reestimate continuously\\
Issues with software or tools & 9 & 5 & 45 & Choose software the team members are familiar with & Hold workshops and view tutorials\\
Customer does not fulfill obligations & 8 & 5 & 40 & Keep customer updated and maintain continuous communication. Set final deadlines for feedback. & Contact supervisor.\\
Illness & 9 & 4 & 36 & Multiple team members work on the same task& Allocate sick member's task to remaining members\\
Unbalanced workload & 5 & 7 & 35 & Coordinate with entire team and log hours & Reallocate tasks\\
Unproductive hours caused by external disruptions & 7 & 5 & 35 & & Find another place to work. Work extra hours to make up for lost time. \\
Team member unavailable & 9 & 3 & 27 & Inform each other of dates we will be unavailable. Good infrastructure for communication and progress reports & Keep in touch with unavailable team member or redistribute tasks\\
%Lars and Beate would possibly have to change to IntelliJ, so this might disappear. Android -> IntelliJ. Back-end->Eclipse/IntelliJ.
%&Eclipse and IntelliJ is incompatible & 7 & 3 & 21 & Evaluate whether use of different IDEs have any negative impact on project & Decide to choose only one of them. \\
External services unavailable & 4 & 7 & 21 & Project not fully dependent of external services. Server is easy to deploy. & Must find new external service to replace the one(s) becoming unavailable. Deploy elsewhere.\\
Spending more time than estimated on discussions & 7 & 3& 21& Include a buffer in our estimations. & Team leader decides if subject is not worth the time or should be rescheduled.\\
Data loss & 2 & 9 & 18 & Use version control system &Restore data from previous versions\\
Customer unavailable & 3 & 6 & 18 & Keep regular contact with customer & Discuss problem within team and contact supervisor\\
Communication failure & 7 & 2 & 14 & Make e-mail-list and exchange contact information. %Be explicit.
 & Check e-mail and phones multiple times a day.\\
Customer requirements exceed project time scope& 4 & 3 & 12 & Continuously revise the workload and prioritize. Set a date for last major changes. & Politely explain that there is not enough time for the changes he suggests. Drop low-priority tasks.\\
Supervisor unavailable & 3 & 4 & 12 & Keep regular contact with supervisor & Discuss problem internally and contact department.\\
%13&Conflict in group & 5 & 2 & 10 & Have in mind that people often misunderstand each other and have an open dialog to solve problem. & Contact supervisor for help.\\
Team member did not complete given assignments & 1 & 7 & 7 & Keep track of the effort and the hour's log public & Contact supervisor and redistribute tasks \\\hline
\caption{Risk analysis table}
\end{longtable}
\rowcolors{0}{}{}

\subsection{The most influential risks}
As explained in the previous section, the team analyzed the possible risks that might occur during the time scope of the project. The risk analysis table gives a certain idea about how much of an impact the team expected a risk to have, but says nothing about which risks that actually occurred and that had a big impact on the project. This section gives a short description of the most influential risks we experienced throughout the project.

\subsubsection{Underestimation of workload}
A prevalent risk to any project is the failure to acknowledge the cost of tasks in the project and the improper allocation of resources that follows such an error. This can also lead to a false sense of what a team can achieve in a given time frame. 

The team experienced this risk to have the biggest negative impact on our workflow. Even though it did not have much effect on the sprint end result, it was the most occurring risk, which added up to be very time consuming. The main reason for this issue was the lack of experience with many of the tasks, but also that we initially did not consider who the tasks were assigned to, which turned out to play an important role in the equation. How the team resolved this issue is further explained in section~\ref{sec:assignEst}.

\subsubsection{Issues with software or tools}
Any team of software developers usually rely on a set of tools and software to help them develop and document a system. Unavailability, lack of experience or improper use of such tools can lead to setbacks in progress.

The team experienced this initially both with the Scrum tool, as explained in section~\ref{sec:improperScrum}, and when setting up the Android development environment.

\subsubsection{Unbalanced workload}
Failure to estimate the cost of tasks, or failure to allocate them wisely, can leave some team members grasping for more work while others are bogged down with either too heavy tasks or a much greater workload.

The team did experience this, but considered it to be more of an opportunity to improve the use of resources and streamline the workflow rather than being a high impact issue. Further details about this issue may be found in section~\ref{sec:unbalancedWorkload}.