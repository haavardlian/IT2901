\subsection{Documentation}

\subsubsection{JavaDoc}
The team used JavaDoc~\cite{javadoc} to document the code. The main advantages of using JavaDoc is that it can be integrated with the Java code and linked directly with classes and methods making it easier for others to further develop and use the code.

\subsubsection{Yodiz}
\todo{move? The team's main requirements in a Scrum tool was that it should be easy to use, effective for project management, had Git integration, and that it could create charts for documentation purposes. With this tool the team have been able to get an overview over the projects progress and the team's performance. 

The team chose Yodiz because it provides free accounts for educational purposes and satisfied the team's requirements.} Yodiz~\cite{yodiz} is an agile project management tool with product backlog management, Kanban Scrum board and issue tracking software.

\subsubsection{Google Drive}
The team decided to use Google Drive~\cite{gdrive} for most of the temporary documentation because it offers an easy way to create documents that is accessible for editing and collaboration simultaneously for the entire team. Here, the team mainly kept documents like meeting agendas, information gathering, input from the customer and the supervisor, product concepts and other documents that waited to get put into the final report. 

In addition, Google Drive serves as an external backup, making it highly unlikely that important documentation will be lost. 


\subsubsection{\LaTeX}
LaTeX is an advanced typesetting system for document production, widely used in
academic institutions. It focuses on allowing the author to focus mainly on the content of the document, and less on the design and document layout.

We chose LaTeX instead of other word processing programs like Microsoft Office, because LaTeX makes it easier to keep track of references and to maintain the appearance of large
documents. In addition, LaTeX provides the ability to break documents into smaller parts and for several team members to simultaneously make changes to the report.

\subsubsection{Maintaining the quality of the report}
Besides producing an app both the team and the customer would be greatly satisfied with, the team also aimed to produce a report of high quality that documents the project's process. To strive for such a report, the team used some tools that proved to be very helpful during the production of the report, as described in table~\ref{tab:reportTools}.

\begin{table}[H]
\rowcolors{0}{darkgray}{lightgray}
\begin{tabular}{|l|p{13.7cm}|}
\hline
\textbf{Tools} & \textbf{What we used them for}\\\hline
\textbackslash todonotes~\cite{todo}&Comment on paragraphs should be rephrased, or whether an illustration was missing. Also provided a list of all the remaining things to do.\\\hline
aspell~\cite{aspell} & Spell checking.\\\hline
BibTex~\cite{bibtex}&To keep track of the citations in the bibliography\\\hline
glossaries~\cite{glossaries}& Provide a brief overview of singular words, with explanations\\\hline
\end{tabular}
\caption{Overview of tools used to maintain report}
\label{tab:reportTools}
\end{table}

\begin{comment}
\subsubsection{\textbackslash todonotes}
Feedback is key when creating a product. The \textbackslash todonotes~\cite{todo} package allowed the team to comment on paragraphs and formulations we wanted to rephrase, whether an illustration was missing, and also gave us a list of all the things we had to do, making it easy to get an overview of the remaining tasks regarding the report.\\

\subsubsection{Spell checking: aspell}
Although manual proofreading cannot be avoided, it is advantageous to have a tool to perform automatic spell checking. Aspell~\cite{aspell} is such a tool.\\

\subsubsection{References and glossary}
To keep track of the citations in the bibliography in the report, we used a LaTeX-package called BibTex~\cite{bibtex}, and for our glossaries, a package called glossaries~\cite{glossaries}.

\end{comment}
