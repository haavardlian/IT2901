\subsection{Documentation}

\subsubsection{JavaDoc}
JavaDoc~\cite{javadoc} was used to document the code. The main advantages of using JavaDoc is that it can be integrated with the Java code and linked directly with classes and methods making it easier for new developers to utilize and modify the code.

\subsubsection{Yodiz}
Yodiz~\cite{yodiz} is an agile project management tool with product backlog management, Kanban Scrum board and issue tracking software. The team used this for distributing tasks among team members, monitor project progress and generating Gantt diagram and sprint burn down charts.

\subsubsection{Google Drive}
Google Drive~\cite{gdrive} was used for temporary documentation because it offers an easy way to create documents accessible for editing and collaborating simultaneously. The team used google drive for documents like meeting agendas, product concepts, and input from the customer and supervisor. 

Google Drive serves as an external backup as well, making loss of data and documentation highly unlikely.

\subsubsection{\LaTeX}
LaTeX~\cite{latex} is an advanced typesetting system for document production, that is widely used in
academic institutions. It focuses on allowing the author to focus mainly on the content of the document, and less on the design and document layout.

The team chose LaTeX instead of other word processing programs like Microsoft Office, because LaTeX makes it easier to keep track of references and to maintain the appearance of large documents. 
LaTeX provides the ability to break documents into smaller parts as well. This makes the content and structure of the report easier to manage and it allows several team members to make changes at the same time.

\subsubsection{Maintaining the quality of the report}
Besides producing an app both the team and the customer would be satisfied with, the team aimed to produce a report of high quality that documents the project's process. To accomplish such a report, the team used some tools that proved to be very helpful during the production of the report, as described in table~\ref{tab:reportTools}.

\begin{table}[H]
\rowcolors{0}{darkgray}{lightgray}
\begin{tabular}{|l|p{13cm}|}
\hline
\textbf{Tools} & \textbf{What we used them for}\\\hline
\textbackslash todonotes~\cite{todo}&Comment on paragraphs should be rephrased, or whether an illustration was missing. Also provided a list of all the remaining things to do.\\\hline
aspell~\cite{aspell} & Spell checking.\\\hline
BibTex~\cite{bibtex}&To keep track of the citations in the bibliography.\\\hline
glossaries~\cite{glossaries}& Provide a brief overview of singular words, with explanations.\\\hline
\end{tabular}
\caption{Overview of tools used to maintain report}
\label{tab:reportTools}
\end{table}

\begin{comment}
\subsubsection{\textbackslash todonotes}
Feedback is key when creating a product. The \textbackslash todonotes~\cite{todo} package allowed the team to comment on paragraphs and formulations we wanted to rephrase, whether an illustration was missing, and also gave us a list of all the things we had to do, making it easy to get an overview of the remaining tasks regarding the report.\\

\subsubsection{Spell checking: aspell}
Although manual proofreading cannot be avoided, it is advantageous to have a tool to perform automatic spell checking. Aspell~\cite{aspell} is such a tool.\\

\subsubsection{References and glossary}
To keep track of the citations in the bibliography in the report, we used a LaTeX-package called BibTex~\cite{bibtex}, and for our glossaries, a package called glossaries~\cite{glossaries}.

\end{comment}
