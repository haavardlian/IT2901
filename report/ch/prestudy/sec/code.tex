\subsection{Code management and versioning}
\subsubsection{GitHub}
The team decided to use Git with GitHub~\cite{github} instead of alternative tools like SVN~\cite{svn} and Mercurial~\cite{mercurial}, because most of us had extensive experience with Git from previous projects. It was also requested by the customer. 

In addition to this, Git offers very good support for team development with advanced functionality like branching and version control to make contributions from several developers easy to manage. GitHub also works as an external backup for the project and some of the documentation, making data loss less likely to occur.

\subsubsection{Development in Android}
For the development in Android, the team used Android Studio~\cite{android-studio}, which is an integrated development environment based on IntelliJ~\cite{intellij}, specifically designed for development on the Android platform.

\subsubsection{Integrated Development Environments (IDEs)}
The main IDEs for Java programming are Eclipse~\cite{eclipse}, NetBeans~\cite{netbeans} and IntelliJ. Most of the team members had experience with Eclipse for both Java and Android development. Despite of this, the team chose Android Studio for the Android development. This was because Gradle and Android Studio contained more relevant functionality than its counterparts, such as a quicker compiler and a practical auto-completion of code.%of Google's decision to spending more resources on further developing Gradle and Android Studio.

The team decided to keep the back-end part optional. All the IDEs have tools for checking the code quality and code conventions, and compatibility between them was only a minor issue that easily could be resolved with Git.

\subsubsection{Maven}
Maven~\cite{maven} is an intelligent project management, construction and application tool, which was used to build and manage the Java-part project. The goal of Maven is to give developers an understanding of the complete state of a development project in the shortest amount of time.

\subsubsection{Gradle}
Gradle~\cite{gradle} is in many ways similar to Maven. It was used to build and manage the Android part of the project. The reason we used Gradle instead of Maven is because it is well integrated with the Android environment. In addition, most of the libraries needed for the Android app is supported through Gradle.

\subsubsection{Code convention}
This project uses Java~\cite{javaconv} and Android's code conventions~\cite{androidconv}, as requested by the customer.
