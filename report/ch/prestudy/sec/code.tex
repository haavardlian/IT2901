\subsection{Code management and versioning}
\subsubsection{GitHub}
The team decided to use Git with GitHub~\cite{github} instead of alternative tools like SVN~\cite{svn} and Mercurial~\cite{mercurial}, because most of the team had experience with Git from previous projects. It was also requested by the customer and used by the CoSSMic project. 

In addition to this, Git offers very good support for team development with advanced functionality like branching and version control to make contributions from several developers easier to manage. GitHub functioned as an external backup for the project, as well, making data loss less likely.

\subsubsection{Integrated Development Environments (IDEs)}
The main IDEs for Java programming are Eclipse~\cite{eclipse}, NetBeans~\cite{netbeans} and IntelliJ~\cite{intellij}. Most of the team members had experience with Eclipse. Despite this, the team chose Android Studio~\cite{android-studio} for development. This was because Gradle and Android Studio contained more relevant functionality than its counterparts, such as a faster compiler, and practical auto-completion of code.

The team decided to keep the back-end part optional. All the IDEs have tools for checking code quality and code conventions, and compatibility between them was only a minor issue that could easily be resolved with Git.

\subsubsection{Maven}
Maven~\cite{maven} is an intelligent project management, construction and application tool, which was used to build and manage the server parts project. The goal of Maven is to give developers an understanding of the complete state of a development project in the shortest time possible.

\subsubsection{Gradle}
Gradle~\cite{gradle} is in many ways similar to Maven. It was used to build and manage the Android part of the project. The reason Gradle was used instead of Maven was because it is well integrated with the Android environment. In addition, most of the libraries needed for the Android app was supported through Gradle.

\subsubsection{Code convention}
This project uses Java~\cite{javaconv} and Android code conventions~\cite{androidconv}, as requested by the customer.
