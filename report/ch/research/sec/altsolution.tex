\section{Existing solutions}
\label{sec:altsolution}
Based on the product requirements the team has worked out, the team did some research on existing products in the 
same market. This resulted in a list of the most relevant products, possibly competing with the team's own application. 
This list contains a brief overview of the functionalities and drawbacks of the existing solutions. 
The team used this study in alternative solutions as an inspiration for functionality and features to implement 
in the project application.

\subsection{Smartly}

Smartly~\cite{smartly} is an application that allows the user to monitor and control things like 
temperature and lighting in the house. It has an overview over total power consumption but not much more. 
Since we want to be able to measure the power usage in each device, this is not that similar to our project. 


\subsection{NTE miniSolo energydisplay}

The NTE miniSolo energydisplay~\cite{nte} is a product measures the total power usage real time and allows 
the user to power off devices to see the effect on the total power consumption. 
It has some interesting features like detailed power consumption and a relation to how much money the power 
usage amounts to. These are features we also want to have in our application. The miniSolo energydisplay is 
however linked to a proprietary device and has no Android application.



\subsection{Theowl}

Theowl~\cite{theowl} is a product that mainly focuses on temperature control. 
It lacks many of the features we want but it has an interesting architecture with remote sensors that 
sends precise data and allow the user to control certain devices. This product has no official support in
Norway and the product is a bit expensive.


\subsection{OpenEnergyMonitor}

OpenEnergyMonitor~\cite{openenergymonitor} is an open source project that allows data collection from power outages. Some of the architecture for collecting data can be an interesting option to consider if improved. The project is open source, and the architecture is somewhat similar to what we imagine using in our project. But with the OpenEnergyMonitor you now have to manually go around your house to collect the data. It is also somewhat hard to set up - so it’s not very user friendly for the average consumer.  This product also has an application for processing, logging and visualizing energy usage. This is features we also want in our product. 



\subsection{Efergy}

Efergy~\cite{efergy} is the product closest to the one the team want to develop. It has nice visual representation of data, is economical comparable, and social integration. The architecture in this product is very interesting as it has support for measuring the power usage of single devices over wireless radio, that communicates with a local receiver. The receiver is connected to a server through the users Internet. What Efergy lacks is the ability to monitor and control private power production and it is somewhat expensive.


\subsection{Comparison of existing solutions}

\begin{table}[H]
\centering
\rowcolors{1}{darkgray}{lightgray}
\begin{tabular}{|l|l|l|l|}
\hline
\textbf{Product} & \textbf{Device control} & \textbf{Social media} & \textbf{Measure production}\\
Smartly & Limited & No  & No \\
Minisolo & Yes & No  & No \\
TheOwl & Yes & No & No \\
OpenEnergyMonitor & Yes & No  & Limited \\
Efergy & Yes & Yes &  No \\\hline
\end{tabular}
\caption{Existing solutions}
\label{tab:existingSolutions}
\end{table}
