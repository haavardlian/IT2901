\chapter{User Manual}

\todo{Should we write the user manual "to" the reader. If the user has a problem, this is where he will get answers, it will be weird to use past tense}

\section{Starting up the app}
\todo{Reference to installation guide?}
Firstly, since this version of the app do not have support for directly connecting the app to power measuring devices in your home, you need to add data. This involves adding all the different devices you want to collect usage for and adding their usage. The usage of a device is added day by day, and if you do not know your past usage it is possible to add an average usage. This can on many devices be found on a note by the power connector on the device.


\label{sec:devices}
\section{Devices}
\subsection{Adding a new device}
\subsection{Editing a device}
%To modify a device already added to your list of devices it is necessary to select the device you want to %modify, by holding your finger over it. (? Over it) When doing this a dialog will appear. Enter the new values %you want to connect to the device you selected. 
\subsection{Deleting a device}


\section{Usage}
The usage tab will give you an overview of your usage. You can see the usage of one or several devices, or you can see your total usage. Here, you can also share your usage with your Facebookfriends.
\subsection{Adding usage}
To add usage to a device, it is important to first have created a device to add usage to. This is done in the device tab. ~\ref{sec:devices}
\subsection{Examine your own usage}


\subsection{Logging in to Facebook}
Logging in to Facebook is done by entering the "login" option in the drawer meny. This will show a login window where you need to enter your username and password. Then press the "Log In" button. 

\section{Comparison}
\subsection{Compare your usage to a friend on Facebook also using Wattitude}
\subsection{Compare your usage based on other people in your area, size of your residence, number of residents or the energy class of your house}

\section{Exchange tips}
\todo{Within the exchange tips tab, you can manage all of your tips. Tips is meant to be advise that will help you save energy around your house, when completed. }

\subsection{Creating your own tip}
To create your own tip, press the green plus symbol in the option bar. This will create a dialog where you must enter a name of your new device, and a description. \todo{Er det noen restriksjoner på hva du kan skrive inn i denne dialogen? Type, må man ha beskrivelse etc.}
\subsection{Adding a tip to your list of tips}
Your list of tips is meant to be a list over the tips you want to perform in you house. 
\subsection{Indicate that you have performed a tip}


\section{Profile}
\subsection{Adding a residence}
