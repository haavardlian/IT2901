\documentclass[12pt]{article}
\usepackage[norsk]{babel}
\usepackage[utf8]{inputenc}
\usepackage{float}
\renewcommand{\arraystretch}{1.5}
\usepackage[left=2.5cm, right=2.5cm,top=2.5cm,bottom=1.5cm,
  includefoot]{geometry}

\title{\textbf{Meeting minutes}\\Team meeting}
\author{}
\date{}
\begin{document}
\maketitle
\pagestyle{empty}
\pagenumbering{gobble}
\vspace{-1cm}

\begin{table}[H]
\begin{tabular}{|p{2cm}p{13cm}|}
\hline
\textbf{Date}&21.01.14\\\hline
\textbf{Present}& Beate Baier Biribakken, Tor-Håkon Bonsaksen, Lars Erik Græsdal-Knutrud, Per Øyvind Kanestrøm, Håvard Holmboe Lian, Pia Karlsen Lindkjølen\\\hline
\textbf{Absent}&\\\hline
\textbf{Meeting agenda}& 1. Meeting times\newline
2. Exchange contact information\newline
3. Development environment\newline
4. Project roles\newline
5. The assignment\newline
6. Other issues\newline
7. Next time
\\\hline
\end{tabular}
\end{table}



\section{Meeting times}
After a brief discussion and having each team member review their time schedule, we concluded that the following times for weekly team meetings would suit the entire team's schedule:\\
- Mondays: 14:15-16:00\\
- Wednesdays: 12:15-14:00\\\\
Should we not get a proper meeting room booked for these times, we will discuss whether we will have a meeting on Sundays as well.


\section{Exchanging contact information}
All the team members write down their phone number. We decide that the primary communication will go through e-mail. Per Øyvind will create an e-mail-list for the team on Google groups(it2901@googlegroups.com).\\\\
\textbf{Phone numbers:}\\
Per Øyvind: 92295543\\
Pia: 97421726\\
Tor-Håkon: 99441554\\
Håvard: 99426621\\
Lars Erik: 97654693\\
Beate: 95444440


\section{Development environment}
We decide to use Git for both the app development and the report. Håvard will create a private repository and the other team members as users on GitHub.


\section{Project roles}
Both Pia and Beate have talked to people who previously have taken the course that suggested it would be a good idea to give each team member an area to be responsible of. After a brief discussion, we ended up with these delegations:\\\\
Customer relations: Pia\\
Project leader: Pia\\
Scrum-master: Per Øyvind\\
Development chief: Tor-Håkon\\
Report editor: Beate\\
Test chief: Håvard\\
Deputy project leader: Lars Erik\\
Secretary: Circulates


\section{The assignment}
\subsection{What do we think the assignment is about?}
Compare yourself to others and see their usage. The competitive aspect. How to save power. Set up measurements to save power, checkbox. Simulator?

Usage data. Typical usage for different housing types, energy marking, age groups, marital status, how many residents that lives in each residence, in which time period the usage is at its greatest, power included in rent. Typical good power saving measurements. Data represented in charts. How much power does my neighbors consume?

\subsection{What should we learn more about before our first meeting with the customer?}
Technical solutions:\\
- Authentication. OAUTH2.0 fb, google\\
- API against electricity suppliers?


\section{Other issues: Questions for the customer}
- How will we retrieve information about power measurement?\\
- Do you have a free room for us to work in?\\
- Did you have a specific target group in mind?\\
- Do you have any specific graphical user interface requirements?\\
- Do you have any usage data we can use in the app?\\
- What do you want to compare? (kwh/month, how much the users have improved, percentagewise?)


\section{For the next time}
- Find a fixed meeting time with the customer. Pia will send a mail to find a time for our first meeting.\\
- Find a tool for Scrum. Suggestion: GitHub\\
- Development tools - Eclipse or IntelliJ?\\
- Set up local Git repository\\
- Choice of rooms, if not assigned one by IDI.\\
- Think about project roles and what they involve.\\

\end{document}
