\chapter{Technical documentation}
The Wattitude server is a RESTful service written in Java and based on the Dropwizard framework. It acts as a HTTP server that receives and answers requests with JSON objects. 
\section{Installation requirements}
The Wattitude server has the following requirements to the environment:
\begin{itemize}
\item Java version 7
\item Apache Maven 
\item Git scm tool
\item MySQL server
\item OS-independent
\end{itemize}
\section{Installation procedure}

\subsection{Installing the app on your phone}
To install Wattitude on your phone you need the apk-file downloaded on your phone and start it. This will automatically launch the apk-installer and voila, you can start using Wattitude!

The following instructions assume that the previously listed requirements are installed and configured correctly. This is especially important to note on Windows systems as they require modification of PATH.
\begin{description}
  \item[Clone repository] \hfill \\
  The systems source code is easily accessible on GitHub. Navigate to desired directory and run "git clone git@github.com:haavardlian/UbiSolar.git"
  \item[Build project] \hfill \\
  The systems server should be build using Apache Maven. Navigate into the "server" directory and build the server by running "mvn package"
  \item[Configure server] \hfill \\
  Traffic between the server and clients run on port 8080. Make sure this port is open in the computer's and network's firewall. Configure the included config.json file, found in the newly built project, to match the settings of your database. Run "java -jar server.jar db migrate config.json" to create the tables needed for running the server. 
  
\end{description}

\section{Running the server}
Running the server is done, simply by navigating into the newly built "server" directory and run "java -jar server-1.0.jar server config.json".

\section{Troubleshooting}